
\chapter{शेल्डन पॉलॉक एवं मीमांसा}\label{chapter1}

\Authorline{आलोक मिश्रा}

\bgroup

\selectdev

\medskip

\centerline{शब्दप्रमाणकाः वयम् यच्छब्दाः आह तदस्माकं प्रमाणम् (महाभाष्यम्\index{Mahabhasya@\textit{Mahābhāṣya}} 4.1.3)}


\section*{भूमिका}

वेद भारतीय संस्कृति की बुनियाद है~~। सत्यसनातन वैदिक धर्म एवं वैदिक संस्कृति का मूल एवं आधार स्तम्भ वेद को विश्व का अत्यन्त प्राचीन एवं आदिवाङ्मय माना जाता है~~। हमारा ज्ञानस्रोत वेद है~~। वेद सर्वज्ञानविज्ञानराशि है~। इसमें ब्रह्मविषयक विचार विज्ञान और शास्त्रशिल्पादि विषय भी है~। मानव जाति के लौकिक एवं पारलौकिक अभ्युदय हेतु सर्गारम्भ में नैसर्गिक रूप से आविर्भूत एवं प्रकाशित होने के फलस्वरूप वेदज्ञान को अनादि नित्य एवं अपौरुषेय कहा जाता है~। परन्तु कितने ही वाक्य परस्पर विरुद्ध से प्रतीत होते है तथा कहीं कहीं वाक्य व्याहतार्थ से दृष्टिगोचर होते है~। इस समस्या के समाधानार्थ वेदवाक्यों के अर्थ निर्धारण के लिये मीमांसाशास्त्र प्रवृत हुआ~।

“पूजितविचारवचनो हि मीमांसाशब्दः”–इन पङ्क्तियों के द्वारा वेद में संभावित संदिग्ध अर्थ का निश्चायक शास्त्र मीमांसा है~। अतः मीमांसा की गणना वेद के उपांगो में की गई~। जैमिनीय\index{Jaimini} मीमांसा\index{purva mimamsa@Pūrva Mīmāṁsā} सूत्रों पर आज उपलब्ध समस्त व्याख्याओं में मूर्धस्थानीय प्रामाणिक भाष्य यदि है तो वह शबर\index{Sabarasvamin@Śabarasvāmin} स्वामी का ही है~। जिसके द्वारा विद्वान् लोग मीमांसा के गूढ रहस्यों को भली–भाँति समझ पाते है~। 

मीमांसा का मुख्य उद्देश्य उन नियमों को बताना है जिनके आधार पर वैदिक वाक्यों एवं कर्मकाण्ड की व्याख्या हो सके~। ब्राह्मणों एवं श्रौतसूत्रों ने भी वेदवाक्यों की उचित व्याख्या को अपना लक्ष्य बनाया था~। मीमांसा उसी का आगे विस्तार कर रही है~। मीमांसा वेद को नित्य, अपौरुषेय एवं स्वतः प्रमाण प्रतिपादित करती है~। परन्तु जिन तर्कों के द्वारा वेद की अपौरुषेयता को मीमांसा शास्त्र में दर्शाया गया है, उन तर्कों को शेल्डन पॉलॉक\index{Pollock, Sheldon} अपने पेपर \enginline{“Language of the Gods in the World of Men”} में निराधार बतलाते हैं~।

अतः अपने इस पेपर मे मैं शेल्डन पॉलॉक की इस बात को रवण्डित करना चाहता हूँ तथा यह स्थापित करना चाहता हूँ कि जिन बातोें को वे निराधार कहते हैं वे बिल्कुल सटीक तथा सत्य है~।


\section*{पूर्वमीमांसा की दृष्टि}

भारतीय संस्कृति यज्ञ\index{yajna@\textit{yajña}} – संस्कृति है~। ऋग्वेद\index{Rg Veda@\textit{Ṛg Veda}} के प्रथम मन्त्र में अग्नि का वर्णन ऋत्विक तथा होता के रूप में किया गया है~। पूर्वतन ऋषियों के द्वारा अग्नि की पूजा की गई है और भविष्य में भी किया जायेगा “अग्निः पूर्वे भिः ऋषिभिः ईड्यः नूतनैः उत”~। पुरुषसूक्त\index{Purusasukta@Puruṣasūkta} में भी कहा गया है “यज्ञेन यज्ञमयजन्त देवाः तानि धर्माणि प्रथमानि आसन्”~। गीता में भी कहा गया है “सहयज्ञाः प्रजाः सृष्ट्वा पुरोवाच–प्रजापतिः” इससे सिद्ध होता है कि धर्म यानी यज्ञ है~। वेद धर्म को सृष्टि का आधार स्तम्भ बताता है~।

\begin{verse}
धर्मो विश्वस्य जगतः प्रतिष्ठा (तैः\index{Taittiriya Aranyaka@\textit{Taittirīya Āraṇyaka}}आ.\enginline{10.63.7})\\ श्रेयो रूपं अत्यसृजत् (शत.ब्रा.\index{Satapatha Brahmana@\textit{Śatapatha Brāhmaṇa}}\enginline{14.4.2.26})\\अन्नाद्भवन्ति भूतानि पर्जन्यादन्नसम्भवः~।\\यज्ञाद्भवति पर्जन्यो यज्ञः कर्मसमुद्भवः ~।। (गी.\index{Bhagavadgita@\textit{Bhagavadgītā}}\enginline{3.14})
\end{verse}

इससे यह प्रतीत होता है कि यज्ञ प्रारम्भ से ही हमारी संस्कृति में समाहित है~। मीमांसा के अनुसार धर्म यानी यज्ञ है अर्थात् लौकिक एवं पारलौकिक अभ्युदय का हेतु एकमात्र धर्म है~। अतः सूत्रकार भगवान् जैमिनि\index{Jaimini} ने मीमांसा शास्त्र का प्रमेय वस्तु निर्धारित करते हुये कहते है “अथातो धर्मजिज्ञासा”, वेदाध्ययन के अनन्तर, वेदाध्ययन समाप्त करने के कारण धर्म–विषयक विचार करना चाहिए~।

मीमांसासूत्र के प्रथमपाद यानी तर्कपाद में सम्यक्तया धर्म का विश्लेषण किया गया है तथापि शेल्डन पॉलॉक\index{Pollock, Sheldon} इन बातो से संतुष्ट नही है~। निराधार ही मीमांसको ने धर्म का प्रयास अलौकिक बताये हैं यह उनका अभिप्राय है~।

\begin{myquote}
\enginline{“First – this is where we encounter the essential \textit{a priori} of Mimamsa – dharma\index{dharma@\textit{dharma}} is stipulatively defined, or rather posited without argument, as a transcendent entity, and so is unknowable by any form of knowledge not itself transcendent”}

~\hfill \enginline{Pollock\index{Pollock, Sheldon} (1989:607)}
\end{myquote}

अतः शेल्डन पॉलॉक\index{Pollock, Sheldon} की इन विचारों को सदोष दर्शाने के लिये सर्वप्रथम हम धर्म के लक्षण तथा धर्म की प्रमाण\index{pramana@\textit{pramāṇa}} को प्रस्तुत करेंगे~।


\section*{धर्म\index{dharma@\textit{ dharma}} का लक्षण}

शबर स्वामी\index{Sabarasvamin@Śabarasvāmin} कहते हैं “स हि निःश्रेयसेन पुरुषं संयुनक्तीति प्रतिजानीमहे~। तदभिधीयते–चोदनालक्षणोऽर्थो धर्मः”~। वह धर्म ही पुरुष को निःश्रेयस से संयुक्त करता है, ऐसी हम प्रतिज्ञा कर रहे है~। उसी को सूत्रकार बता रहे है– चोदनालक्षणोऽर्थोे धर्मः – प्रवर्तक शब्द को ‘चोदना'\index{codana@\textit{codanā}} शब्द से कहा जाता है~। प्रवर्तक शब्द को ही ‘विधायक' कहते है~। यद्यपि ‘विधि' सम्पूर्ण वेद का एकदेश है तथापि वह प्रधान है~। इसलिये सूत्रकार ने “प्राधान्येन व्यपदेशा भवन्ति” इस नियम के अनुसार उक्त धर्मलक्षण में ‘चोदना' शब्द का प्रयोग ‘सम्पूर्ण वेद' के अर्थ में किया है, केवल विधि के अर्थ में नहीं~। तब सूत्रार्थ हुआ–वेदबोधित होकर जो अनर्थ से सम्बन्धित न हो, वह ‘धर्म' है~। क्रिया के प्रवर्तक वचन (शब्द) को ‘चोदना' शब्द से कहते है~। “आचार्य के द्वारा प्रेरित (प्रवर्तित) होता हुआ मैं कर रहा हूँ” इस प्रकार लौकिक व्यवहार प्रचलित है~। जिससे कोई वस्तु (पदार्थ) जानी जाती है, उसे ‘लक्षण' कहते है~। “धूम अग्नि का लक्षण है” ऐसा लोग कहा करते है~। उस चोदना (विधि से सम्बन्धित अंश–चतुष्टयात्मक सम्पूर्ण वेद) से जो श्रेयःसाधन (अनर्थ सम्बन्ध शून्य) अर्थ बोधित किया जाता है वही (अर्थरूपधर्म) पुरुष को निःश्रेयस से जोड़ता है, ऐसी प्रतिज्ञा हम कर रहे है~।

“चोदना हि भूतं भवन्तं भविष्यन्तं सूक्ष्मं व्यवहितं विप्रकृष्टम् इत्येवंजातीयकमर्थं शक्नोत्य\-वगमयितुं, नान्यत् किञ्च नेन्द्रियम्” यह चोदना (वेद) ही निश्चित रूप से भूत, वर्तमान, भविष्यत् सूक्ष्म व्यवहित और दूर स्थित सभी प्रकार के अर्थ को बताने में समर्थ है~। इस चोदना शब्द (वेद) के अतिरिक्त अन्य कोई अनुमानादि प्रमाण इतना समर्थ नहीं है इतना अभिप्राय ‘नान्यत् किञ्च' से प्रकट किया गया है~। अर्थात् शब्द प्रमाण में ही तादृश अर्थ के बोधन कराने का सामर्थ्य है, तद्व्यतिरिक्त प्रमाणसामान्य में वह सामर्थ्य नहीं है~। अपने इस कथन में हेतु बताते है–‘नेन्द्रियम्'–क्योंकि इन्द्रिय (प्रत्यक्ष\index{pratyaksa@\textit{pratyakṣa}}) ही जब असमर्थ होंगे तब अन्य प्रमाण\index{pramana@\textit{pramāṇa}} तन्मूलक (प्रत्यक्षमूलक\index{pratyaksamula@\textit{pratyakṣamūla}}) होने से कैसे समर्थ होंगे ~। अर्थात् शब्दातिरिक्त सभी प्रमाण सब कुछ बोधन कराने में असमर्थ है~। चोदना से होने वाला ज्ञान, किसी भी काल में, किसी भी पुरुष को, किसी भी अवस्था में, किसी भी देश में विपरीत नही होता, इसलिये चोदना\index{codana@\textit{codanā}} (शब्द) से उत्पन्न ज्ञान को सत्य कहना ही होगा~। 

लौकिक वाक्य (पुरुष–वाक्य) प्रमाण तथा अप्रमाण दोनो प्रकार के उपलब्ध होते है~। जो लौकिक वचन (व्यवहार में लोगो द्वारा उच्चारित शब्द) है, वह यदि आप्त (प्रत्ययित–विश्वस्त) पुरुष के द्वारा उच्चरित हो, अथवा इन्द्रिय के द्वारा जिसे जाना जा सके) हो तो वह सत्य (अवितथ) ही है~। और यदि अनाप्त (अविश्वस्त, अप्रामाणिक) पुरुष के द्वारा उच्चारित हो अथवा अनिन्द्रिय विषय वाला (इन्द्रिय के द्वारा जिसे न जाना जा सके) हो, तो वह (शब्द) पुरुष की दूषित बुद्धि से उत्पन्न होने के कारण अप्रमाण है~। वैदिक शब्द के बिना धर्म\index{dharma@\textit{dharma}} का ज्ञान होना किसी भी पुरुष को सम्भव नहीं है~। यदि यह कहें कि अन्य किसी पुरुष के वचन से धर्म का ज्ञान हुआ हो तो वह भी उसी के (पुरुष की दूषित बुद्धि– से उत्पन्न के) समान होगा~। अतः इस वाक्य के पुरुष बुद्धिप्रभव तथा अनिन्द्रियविषयक वस्तुओं (अर्थो–पदार्थो) में पौरुषेय शब्द का प्रमाण नही माना जाता ~। जैसे–जन्मान्ध पुरुषों का वचन ‘रूप विशेष' के सम्बन्ध में प्रमाण नहीं माना जाता है~।

सामान्यतो दृष्टानुमान से भी वेदवचन को मिथ्या समझना ठीक नहीं है, क्योंकि पौरुषेय\index{pauruseya@\textit{pauruṣeya}} वचन से वेदवचन भिन्न है~। किसी अन्य के मिथ्या होने पर किसी अन्य को भी मिथ्या समझ लेना उचित नहीं है, क्योंकि वे दोनो एक–दूसरे से अन्य हैं~। अतः अन्य होने से ही वे एक–दूसरे जैसे नहीं रहते~। श्यामल वर्ण के देवदत्त को देखकर मनुष्य की समानता के आधार पर यज्ञदत्त को भी श्यामल वर्ण का समझ लेना उचित नहीं माना जाता, क्योंकि वे दोनो परस्पर भिन्न व्यक्ति है~। तात्पर्य यह है कि सामान्यतोदृष्टानुमान के द्वारा वेदवचन को मिथ्या नहीं कह सकते और वेदवचन के मिथ्या न होने पर एक कारण और वेदवचन से होने वाला ज्ञान प्रत्यक्ष है~। अतः प्रत्यक्ष से विरोध करके अनुमान का उदय होना सम्भव ही नहीं है~।

मीमांसक के मत में ज्ञान अनुमेय है~। अतः आपका अनुमान प्रत्यक्ष से बाधित हो जाता है~। एवञ्च प्रत्यक्षविरोधी अनुमान\index{anumana@\textit{anumāna}} को प्रमाण से बोधित अर्थ श्रेयस्कर है~। शबर\index{Sabarasvamin@Śabarasvāmin} स्वामी कहते है–“तस्माच्चोदनालक्षणोऽर्थः श्रेयस्करः”~। उपक्रम और उपसंहार का ऐक्य होना चाहिए यह नियम है~। भाष्यकार ने “को धर्मः, कथं लक्षणः”–इस प्रकार धर्म शब्द से उपक्रम किया है, तब उसी शब्द से उपसंहार भी करना चाहिए था, किन्तु श्रेयस्कर शब्द से उपसंहार करने में भाष्यकार का अभिप्राय क्या होगा? यह जिज्ञासा होनी स्वाभाविक है~। विचार करने पर, भाष्यकार का अभिप्राय यह प्रतीत होता है कि “पदार्थधर्मः\index{padarthadharma@\textit{padārthadharma}}”, “पक्षधर्मः\index{paksadharma@\textit{pakṣadharma}}” इत्यादि प्रयोग के अनुसार “वृत्तिमत्” के अर्थ में भी धर्म शब्द का प्रयोग होता है~। अतः धर्म शब्द के अनेकार्थक होने से किस अर्थ में धर्म शब्द को यहाँ पर लिया जाय? इस शंका के समाधानार्थ ‘श्रेयस्कर शब्द से उपसंहार किया गया है अर्थात् जो श्रेयस्कर हो वही धर्म शब्द से यहा ग्राह्य है, अन्य नहीं ~। जो निःश्रेयस\index{moksa@\textit{mokṣa}} से पुरुष को संयुक्त करता है उसे लोग धर्म शब्द से कहते है~। यह व्यवहार केवल लोक में ही नहीं अपितु वेद में भी इसी प्रकार का व्यवहार किया गया है~। “यज्ञेन\index{yajna@\textit{yajña}} यज्ञमयजन्तदेवास्तानि धर्माणि प्रथमान्यासन्” (ऋ.\index{Rg Veda@\textit{Ṛg Veda}}\enginline{9/90/16}, शु.य.वा.सं.\index{Yajurveda@\textit{Yajurveda}}\enginline{39/16}) इन्द्रिय देवताओं ने ज्योतिष्टोमसंज्ञक यज्ञ से यज्ञ पुरुष वासुदेव का यथाविधि यजन किया, इस कारण वे प्रथम (मुख्य) यजनरूप धर्म हुए~। उक्त वेदवाक्य में भी “यज्” धातु के वाच्य (मुख्य) अर्थ याग को ही ‘धर्म' शब्द से कहा गया है~।

\newpage

अतः उपसंहार करते हुये यह कह सकते है कि यह सूत्र वाक्यद्वयरूप है, अर्थात् इस एक सूत्र में दो वाक्य अन्तर्निहित है~। अतः सूत्रान्तर्गत रहने वाले दो वाक्यो से ही दो अर्थो की प्रतीति हो रही है~। इससे यह स्पष्ट हुआ कि अर्थत्व विशिष्ट होकर ही चोदनालक्षण\index{codana@\textit{codanā}} हो वह धर्म\index{dharma@\textit{dharma}} है, और अर्थ भी चोदनालक्षणत्वविशिष्ट होकर ही धर्म है~।

प्रकारान्तर से भी द्वितीय सूत्र का अर्थ बताया जा सकता है~। “अर्थस्य सतः यद्धर्मत्वं तच्चोदनालक्षणस्य” इत्युच्यते~। इस द्वितीय सूत्र का ‘यो धर्मः स चोदनालक्षणः' जो धर्म है, वह चोदनालक्षण (शब्दप्रमाण) है–यही अर्थ करना होगा~।

\vskip 5pt

\section*{धर्म\index{dharma@\textit{dharma}} में प्रमाण\index{pramana@\textit{pramāṇa}} की परीक्षा}

अपने पेपर \enginline{“Mimamsa and the Problem of History in Traditional India”} में शेल्डन पॉलॉक\index{Pollock, Sheldon} यह कहते हैं कि एक प्रमाण से जानी हुई वस्तु अन्य प्रमाणों से नहीं जानी जा सकती~। ऐसा केवल मीमांसक\index{Mimamsaka@Mīmāṁsaka} कहते है~।

\vskip 4pt

\begin{myquote}
\enginline{“Second – and this is the basic epistemological position of Mimamsa: all cognitions must be accepted as true unless and until they are falsified by other cognitions.”}

~\hfill \enginline{Pollock\index{Pollock, Sheldon} (1989:607)}
\end{myquote}

\vskip 4pt

पॉलॉक की यह बात हास्यास्पद है~। भारतीय षड् दर्शनों में विस्तार से दर्शाया है कि एक प्रमाण से ज्ञात हुआ वस्तु अन्य प्रमाणों से नहीं ज्ञात हो सकता यह बात केवल मीमांसा की नहीं है~। न्याय में प्रमाण शब्द का निर्वचन ‘प्रमाकरणं प्रमाणम्' ‘असाधारणं कारण् करणम्' ~। ‘किं नाम असाधारणत्वम्? लक्ष्यताऽवच्छेदकसमनियतत्वम् असाधारणत्वम्' मीमांसको ने धर्म के ज्ञान होने में असाधारण निमित्त एकमात्र चोदना (वैदिक शब्द) को बताया है~।

\vskip 2pt

अर्थात् धर्म का एकमात्र शब्दप्रमाण से ही ज्ञान हो सकता है~। वह कथन केवल प्रतिज्ञारूप से ही था~। युक्ति से उसे सिद्ध नहीं किया था, किन्तु अब हम उस धर्म के निमित्त की परीक्षा करेंगे–क्या चोदना\index{codana@\textit{codanā}} ही धर्मज्ञान में निमित्त है, अथवा तदतिरिक्त कोई अन्य प्रमाण भी धर्मज्ञान कराने में निमित्त है? परीक्षण करने से पूर्व यह निश्चय नहीं हो पा रहा है कि चोदना से लक्षित होने वाला अर्थ ही धर्म है~। परीक्षण करने पर धर्म के निमित्त का ज्ञान अनायास हो जायेगा उसी के लिये परीक्षा करने की प्रतिज्ञा करते हुये सूत्रकार जैमिनि मुनि कहते है–“तदुच्यते सत्सम्प्रयोगे पुरुषस्येन्द्रियाणां बुद्धिजन्म तत् प्रत्यक्षम् अनिमित्तं विद्यमानोपलम्भनत्वात्” – इन्द्रियों का विद्यमान वस्तु के साथ सम्बन्ध होने पर पुरुष के ज्ञान की उत्पत्ति होती है~। उसे प्रत्यक्ष कहते है~। वह प्रत्यक्ष विद्यमान वस्तु का ज्ञान कराता है इसलिए धर्म\index{dharma@\textit{dharma}} का ज्ञान कराने में निमित्त नहीं हो सकता क्योंकि प्रत्यक्षात्मक ज्ञान की उत्पत्ति के समय धर्म की सत्ता नहीं है~। वह तो भविष्यत्कालिक है~।

‘तस्य निमित्तपरीष्टिः' इस सूत्र में कर्तव्य रूप से प्रतिज्ञात परीक्षण का आरम्भ किया जा रहा है~। चोदनासूत्र\index{codana@\textit{codanā}} द्वारा प्रदर्शित ‘चोदनैव धर्मे प्रमाणम्'\index{pramana@\textit{pramāṇa}} इत्याकारक प्रतिज्ञा उचित नहीं प्रतीत हो रही है~। इस प्रकार की आक्षेप की प्राप्ति होने पर प्रकृत सूत्र से प्रतिपादित सिद्धान्त बताया जा रहा है~। धर्मज्ञान के प्रति ‘प्रत्यक्ष' अनिमित्त है, अर्थात निमित्त नहीं है~। इस पर प्रश्न किया कि धर्म\index{dharma@\textit{dharma}} के प्रति निमित्त न हो सकने में क्या कारण है? वह प्रत्यक्ष एवंलक्षणक है, अर्थात् धर्म के प्रति अनिमित्तता जिस लि‘ से लक्षित होती है, उस लक्षण वाला वह प्रत्यक्ष है~। उसी लिङ्ग को ‘सत्सम्प्रयोगे पुरुषस्येन्द्रियाणां बुद्धिजन्म तत्प्रत्यक्षम्' ‘सतीन्द्रियार्थ सम्बन्धे या पुरुषस्य बुद्धिर्जायते तत्प्रत्यक्षम्' सूत्रस्थ' इन्द्रियाणाम्' इस पद का ‘सत्सम्प्रयोगे' इस पद के ‘सम्प्रयोगे' के साथ और ‘पुरुषस्य' पद का ‘बुद्धिजन्म' पद के ‘बुद्धि' के साथ सम्बन्ध है~। ‘सत्सम्प्रयोगे' इस पद में ‘संश्चासौ सम्प्रयोगश्च' सत्सम्प्रयोगः~। ऐसा कर्मधारय समास करना है~। तथा च इन्द्रियों का अर्थ (विषय, वस्तु, पदार्थ) के साथ सम्बन्ध होने पर ‘पुरुषस्य बुद्धिजन्म' के अवयवार्थ को भाष्यकार स्पष्ट करते है– ‘या पुुरुषस्य बुद्धिर्जायते'~। तथा च ‘जन्म' शब्द कर्तृवाचक होता हुआ बुद्धि शब्द का समानाधिकरण प्रदर्शित किया है~। तब अर्थ यह हुआ – पुरुष की उत्पन्न होने वाली (जायमाना जो बुद्धि (ज्ञान) वह (ज्ञान) प्रत्यक्ष कहलाता है~। जबकि उक्त रीति से सत् सम्प्रयोगज है, अतः उसकी विद्यमानोपलम्भनता है, अर्थात् वह विद्यमान–वस्तु का उपलम्भक है~। विद्यमानोपलम्भनत्व की सिद्धि के लिये सत्सम्प्रयोगजत्व को बताया गया है और धर्म के प्रति प्रत्यक्ष के अनिमित्त होने में ‘विद्यमानोपलम्भनत्व' ही प्रयोजक है~। तथा च सूत्र और भाष्य के द्वारा तीन प्रयोग प्रदर्शित किये गये है–तथा हि–

\begin{enumerate}
\item “प्रत्यक्षं धर्माऽधर्मागोचरं विद्यमानोपलम्भनत्वात्”

 \item “प्रत्यक्षं विद्यमानोपलम्भनं (विद्यमानार्थोपलब्धिरूपं) वर्तमानेन्द्रियार्थ–संयोगजन्यत्वात्”

 \item “प्रत्यक्षं सत्सम्प्रयोगजं प्रत्यक्षत्वात्”~।

\end{enumerate}

विद्यमान अर्थ के अवगाहक प्रत्यक्ष की धर्माऽधर्मागोचरता का उपपादन ‘भविष्यंश्चैषोऽर्थो न ज्ञानकालेऽस्तीति' के द्वारा भाष्यकार कर रहे है~। स्वकालिक अर्थविषयक प्रत्यक्ष में स्वकाल में अविद्यमान धर्माऽधर्मविषयकत्व का होना सम्भव नहीं है~। ‘विद्यमानोपलम्भनत्वात्' इस सूत्रावयव के अर्थ को ‘सतश्चैतदुपलम्भनं नासतः' भाष्य से स्पष्ट किया गया है~। ‘अतः प्रत्यक्षमनिमित्तम्' इस भाष्य से सबका निष्कर्ष बताया कि धर्मज्ञान के प्रति' प्रत्यक्ष प्रमाण'\index{pramana@\textit{pramāṇa}} निमित्त नहीं है~।

उपर्युक्त अनुमान–प्रयोगों में योगिप्रत्यक्ष को ही ‘पक्ष' रखा गया है और अस्मदादिप्रत्यक्ष को ‘दृष्टान्त' किया गया है~। उसी से साध्य का साधन करने के कारण ‘दृष्टान्तासिद्ध' नहीं है और न ही सिद्धसाध्यता है~।

इस रीति से यागादि को फलसाधनत्वरूप से ही ‘धर्म'\index{dharma@\textit{dharma}} माना गया है~। निष्पन्न अवस्था में याग का स्वरूप प्रत्यक्ष रहने पर भी फलसाधनत्व जो याग का विशिष्ट रूप है उसका प्रत्यक्ष होना कभी भी सम्भव नहीं है~। क्योंकि विशेषणीभूत फल कालान्तरभावी है, और ‘अपूर्व' तो स्वभावतः ही अप्रत्यक्ष है~। यह समझना चाहिए~। बुद्धि (ज्ञान) अथवा बुद्धिजन्य हानोपादान बुद्धि अथवा इन्द्रिय और अर्थ का सम्बन्ध (सन्निकर्ष) – इनमें से किसी एक को प्रत्यक्ष कहते है–इस अवधारण (निर्णय/निश्चय) के लिये यह सूत्र नहीं है~। 

तात्पर्य यह है कि इन्द्रिय आदि, या इन्द्रियार्थ संयोग आदि, अथवा तत्तदर्थकविषय बुद्धि, अथवा तज्जन्य हानोपादानादि बुद्धि, का प्रमाण\index{pramana@\textit{pramāṇa}} फल भाव के विषय में अनादर सूचित किया गया है~। ‘इन्द्रिय और अर्थ का सम्बन्ध होने पर ही प्रत्यक्ष होता है, और इन्द्रियार्थ का सम्बन्ध न होने पर प्रत्यक्ष नहीं होता~। प्रत्यक्ष का धर्म के प्रति अप्रमाण्य–समर्थन करने से ही तत्पूर्वक होने वाले अनुमान, उपमान, अर्थापत्ति प्रमाणों का सुतरां अप्रामाण्य प्रदर्शित हो जाता है~। अर्थात् अनुमान–उपमान आदि अन्य प्रमाण प्रत्यक्षपूर्वक हुआ करते है~। इसलियें उन प्रमाणों को भी धर्म के प्रति कारण नहीं समझना चाहिए, क्योंकि अन्य प्रमाण प्रत्यक्ष के ही आश्रित रहते है~। जब प्रत्यक्ष ही धर्मज्ञान कराने में समर्थ नहीं है तो उसके आश्रित रहने वाले अन्य प्रमाण उसका ज्ञान कराने में कैसे समर्थ हो सकेंगे~। 

जगद्वैचित्र्यान्यथानुपपत्तिरूप अर्थापत्ति के द्वारा दृष्ट कारणभिन्न अदृष्ट कारण का आक्षेप हो सकने पर भी ‘इदमस्य साधनम्'–यह इसका साधन है–इस प्रकार विशेष रूप से आक्षेप न कर पाने के कारण उसका भी अप्रामाण्य स्पष्ट ही है~। शबर\index{Sabarasvamin@Śabarasvāmin} स्वामी कहते है ‘अभावोऽपि नास्ति~। यतः “औत्पत्तिकस्तु शब्दस्यार्थेन सम्बन्धस्तस्य ज्ञानमुपदेशोऽव्यतिरेकश्चार्थे–अनुपलब्धे तत्प्रमाणं बादरायणस्याऽनपेक्षत्वात्”~। शब्द का अर्थ के साथ सम्बन्ध औत्पत्तिक (स्वाभाविक) है, उस धर्म का ज्ञान, साधन–अर्थात् ज्ञापक उपदेश (विधिघटित वाक्य है~। उसका कभी अव्यतिरेक विपर्यय) नहीं होता है~। इसलिये वह विधिघटित वाक्य अनुपलब्ध अर्थ में भी प्रमाण है~।

बादरायण आचार्य के मत में भी, “अनपेक्षत्वात्”–प्रत्ययान्तर की अथवा पुरुषान्तर की अपेक्षा न होने से वह स्वतः प्रमाण\index{pramana@\textit{pramāṇa}} है~। भाष्यकार सूत्रावयव ‘औत्पत्तिक' शब्द का अर्थ ‘नित्य' बता रहे हैं~। औत्पत्तिक–नित्य इस प्रकार अर्थ करनें में उपपत्ति बताते है कि ‘उत्पत्तिर्हि भाव उच्यते लक्षणया'~। ‘औत्पत्तिक शब्द की व्युत्पत्ति से यद्यपि नित्य अर्थ प्राप्त नहीं हो रहा है तथापि लक्षणा से नित्य अर्थ कि प्राप्ती हो जाती है~। इसी बात को कहा है कि ‘उत्पत्ति' शब्द लक्षणा से भाव (सत्ता) स्वभाव अर्थ को बतलाता है~। तथा च औत्पत्तिक का अर्थ है– ‘स्वाभाविक'~। इसी अर्थ को और अधिक स्पष्ट करते है– “अवियुक्तः – शब्दार्थयोर्भावः सम्बन्धो नोत्पन्नयोः पश्चात् सम्बन्धः”, शब्द और अर्थ का सम्बन्ध अवियुक्त ‘भाव' है तथा च ‘औत्पत्तिक' शब्द से ‘नित्यत्व' अभिप्रेत है~। एवञ्च शब्द का अर्थ के साथ जो प्रत्याय्य–प्रत्यायकलक्षण सम्बन्ध है, वह नित्य है~। यहाँ भाष्यकार ने औत्पत्तिक शब्द के द्वारा कारणगत दोष से होने वाले अप्रामाण्य का निराकरण कर दिया है~। यदि शब्दार्थ का सम्बन्ध कृतक होता तो तद्द्वारा भी पुरुष दोष के प्रवेश की आशंका से अप्रामाण्य होता है, लेकिन यह कुछ नहीं है~। 

शब्दार्थ सम्बन्ध स्वाभाविक रहने से पुरुषाधीनता नहीं है~। उत्पन्न हुए शब्द और अर्थ का पीेछे से किसी के द्वारा सम्बन्ध जोड़ा नही गया है~। उन दोनों का सम्बन्ध तो औत्पत्तिक (स्वाभाविक, नित्य) है~। वहीं सम्बन्ध, प्रत्यक्षादि प्रमाणों\index{pramana@\textit{pramāṇa}} से अनवगत (अज्ञात) अग्निहोत्रादि रूप धर्म का निमित्त है~। इससे निष्कर्ष यह निकला कि लोकव्यवहार में प्रमाणान्तरमूलक जो हो उसका प्रामाण्य और प्रमाणान्तरमूलक न हो उसका अप्रामाण्य यद्यपि देखा जाता है तथापि ‘प्रामाण्य अन्यसापेक्ष नहीं है~। अपितु स्वतः ही है~। अनाप्तवाक्य का अप्रामाण्य मूलाभाव के कारण नहीं है जिससे आप्त वाक्य का प्रामाण्य मूल के अधीन कहा जा सके~। अनाप्त वाक्य का अप्रामाण्य तो उसका दूषित मूल होने के कारण~। शब्द के दूषित हो जाने से उसके अपने स्वाभाविक प्रामाण्य का बाध हो जाता है~। अपौरुषेय वेद यद्यपि आप्तप्रणीत नहीं है तथापि प्रामाण्य का प्रयोजक आप्त–प्रणीतत्व न होने के कारण और अनाप्तस्पर्शनिमित्त दोष भी न होने के कारण उसका (वेद का) प्रामाण्य अबाधित ही बना रहता है~। शब्दार्थ सम्बन्ध को औत्पत्तिक सिद्ध करने की आवश्यकता इसलिये हुई कि पुरुष का सम्बन्ध तीन प्रकार से होने की सम्भावना की जा सकती है – \enginline{1}–पद–पदार्थ सम्बन्ध के द्वारा \enginline{2}–वाक्य–वाक्यार्थ सम्बन्ध के द्वारा \enginline{3}–रामायण–महाभारतादि ग्रन्थ के समान ही पौरुषेय होने से~। किंतु यहा (वेद में) तीनों ही नही है, क्योंकि सूत्रकार जैमिनि मुनि इस सूत्र के द्वारा पद–पदार्थ सम्बन्ध को औत्पत्तिक (नित्य, स्वाभाविक) शब्द से बता रहे है~। वाक्यार्थज्ञान पदार्थमूलक होता है और वेद अपौरुषेय है, यह आगे बताया जायेगा~। 

अतः स्वतः प्रमाणाभूत चोदनात्मक\index{codana@\textit{codanā}} शब्द के अप्रामाण्य में कारणदोषज्ञानरूप हेतु की सम्भावना किसी तरह भी नहीं की जा सकती है~। सूत्र के अवयवभूत ‘अर्थेऽनुपलब्धे' का विवरण ‘औत्पत्तिकस्तु–शब्दस्यार्थेन सम्बन्धस्तस्याग्निहोत्रादिलक्षणस्य धर्मस्य\index{dharma@\textit{dharma}} निमितं–प्रत्यक्षादिभिरनवगतस्य~। कथम् ~। उपदेशो हि भवति~। उपदेश इति विशिष्टस्य शब्दस्योच्चारणम्' भाष्य के द्वारा किया गया है~। इस प्रत्यक्षाद्यनवगतार्थत्व के बताने में अनुवादरूप अप्रामाण्य का निराकरण हो जाता है~। यह जो कहा गया था कि ‘शब्दार्थ का स्वाभाविक सम्बन्ध, प्रत्यक्षादि प्रमाणों से अज्ञात अग्निहोत्रादि– रूप धर्म का निमित्त हैं उसमें हेतु बताने की इच्छा से भाष्यकार ‘कथम्' शब्द से प्रश्न कर रहे है अर्थात् उक्त सम्बन्ध अग्निहोत्रादि का निमित्त किस प्रकार है? उसके निमित्त होने में हेतु यह है कि ‘उपदेशो हि भवति' ~। उपदेश इति विशिष्टस्य शब्दस्योच्चारणम्~। श्रेयः साधनभूत उस अग्निहोत्रादि धर्म का ज्ञान कराने वाला ‘अग्निहोत्रंजुहुयात् स्वर्गकामः' इत्यादि विधिवाक्य का उपदेश है~। ‘उपदेश' शब्द का अर्थ है–विशिष्ट शब्द का उच्चारण, अर्थात् श्रेयःसाधनत्वादि अर्थ का प्रतिपादक होने से अभ्यर्हित शब्द (अग्निहोत्रं जुहुयात् स्वर्गकामः) का उच्चारण किया गया है~। इस कथन से विधिवाक्य में ज्ञानानुत्पादकत्व अप्रामाण्य का भी निरसन हो जाता है~। उसी तरह ‘अव्यतिरेकश्च ज्ञानस्य' विधिवाक्य (चोदनाशब्द\index{codana@\textit{codanā}}) से होने वाले ज्ञान का कभी व्यतिरेक नहीं होता है~। इस भाष्य से बाधक ज्ञानरूप अप्रामाण्य कारण भी नहीं है, यह बता दिया गया है~। ‘अव्यतिरेक' शब्द का अर्थ बतलाने के लिये कहते है कि – न हि तदुत्पन्नं ज्ञानं विपर्येति'–विधिवाक्य से उत्पन्न हुआ कभी भी विपर्यय (मिथ्यात्व) को प्राप्त नहीं होता~। अर्थात् अग्निहोत्रादि वाक्य से हुये ज्ञान का बाधक अन्य ज्ञान (प्रत्ययान्तर) (न अग्निहोत्रहोमः स्वर्गसाधनम् इत्याकारक ज्ञानम्) कदापि किसी भी प्रमाण से उत्पन्न नहीं होता है~। 

‘यच्च नाम ज्ञानं न विपर्येति, न तच्छक्यते वक्तुं नैतदेवमिति' – इस भाष्य से यह बताया जा रहा है कि विपर्यय ज्ञानरहित चोदनाजन्य ज्ञान का अप्रामाण्य कहना कथमपि शक्य नही है~। अर्थात् वह वैसा नही है यह नही कह सकते ~। बाधक–ज्ञान से रहित विधि–वाक्य–जनित–ज्ञान का प्रामाण्य भी स्वीकार न करने पर अनिष्ट–प्रसंग होगा~। इस आशय को “यथा भवति–यथा विज्ञायते न तथा भवति~। यथा एतन्न विज्ञायते तथा एतदिति~। अस्य हृदये अन्यत् वाचि स्यात्~। एवं वदतो विरुद्धमिदं गम्यते”~।

अस्ति नास्ति वा इति~। “उक्त भाष्य से स्पष्ट किया गया है~। यदि जैसा ज्ञान होता है (जैसा जाना जाता है) वैसा नही होता है और जैसा ज्ञान नहीं होता है, उसके विपरीत वैसा यदि जाना जाता है, यह स्वीकार करने पर इसके हृदय (मन) में अन्य बात है~। वाणी में अन्य बात है यह कहना होगा~। मन में कुछ और बाहर (शब्द में) कुछ कहने वाले का कथन विरुद्ध समझा जाता है~। “अर्थात् है” और “नही है” यह कथन नितान्त विरुद्ध है~। निष्कर्ष यह है कि प्रतीयमान अर्थ का परित्याग कर अप्रतीयमान अर्थ की कल्पना करना, मन में कुछ और बाहर कुछ (“अन्यत् अस्य हृदये अन्यद् वाचि”) रहने से प्रतारक मनुष्य के वचन के तुल्य ही वेदवाक्य) प्रमाण\index{pramana@\textit{pramāṇa}} है, क्योंकि वह निरपेक्ष है ~। अर्थात् प्रमाणान्तर की अपेक्षा न रखने के कारण वह स्वतः प्रमाण है~। तात्पर्य यह है कि स्वतःप्रमाणभूत वेदवचन को अप्रमाण कहने के लिए कारण दोष, बाधक–ज्ञान, अनुवादकत्व और ज्ञानाऽनुत्पादकत्वादि कारणों में से कोई एक भी कारण नही है~। अतः उस वैदिक शब्द (विधिवाक्य) का निजी स्वारासिक प्रामाण्य निर्बाध रूप से सिद्ध हो जाता है~।


\section*{वेद की अपौरुषेयता}

अनुविद्धमिव ज्ञानं सर्वं शब्देन भासते (भर्तृ.वा.\enginline{1.123})

शेल्डन पॉलॉक\index{Pollock, Sheldon} यह कहता है कि मीमांसको ने जिस प्रकार वेद की अपौरुषेयता को सिद्ध किया है, वह समीचीन नही है~।

\begin{myquote}
\enginline{“It is argued that the Vedas are transcendent by reason of their anonymity. Had they been composed by men, albeit long ago, there is no reason why the memory of these composers should not have been preserved to us. Those men who are named in association with particular recessions, books, hymns of the Vedas–Kathaka, for example, or Paippaladaka are not to be regarded as the authors but simply as scholars specializing in the transmission or exposition of the texts in question. Texts for which no authors can be identified have no authors, and this applies to the Vedas and to the Vedas alone”.}

~\hfill \enginline{Pollock\index{Pollock, Sheldon} (1989:608)}
\end{myquote}

\begin{myquote}
\enginline{“Other arguments are offered, such as those based on the language and style of the Vedas, for example. In answer to a \textit{purvapaksa} averring that (whereas words may be external) sentences can only be composed by men, Śabara\index{Sabarasvamin@Śabarasvāmin} claims the argument has been refuted by the anonymity of the Vedic texts, when that has yet to be proven”}\hfill \enginline{Pollock (1989:608)}
\end{myquote}

\begin{myquote}
\enginline{“The claim for the beginninglessness of Vedic recitation is nowhere clearly sustained in the Bhasya...”}\hfill \enginline{Pollock (1989:608)}
\end{myquote}

\begin{myquote}
\enginline{“A final example is the argument advanced by Śabara that I find to be patently circular. The truth of the content of the Vedas depends on their being \textit{apauruṣeya\index{apauruseya@\textit{apauruṣeya}}; apauruseyatva}, however, is made to depend on the fact that they discuss metaphysical matters – i.e., to depend on the truth of their content...”}

~\hfill \enginline{Pollock (1989:608)}
\end{myquote}

\begin{myquote}
\enginline{“If the Veda is external, it cannot communicate information about non–external things; nor can it do so even if it is not external; for then no absolute authority would attach to any of its communications?”}\hfill \enginline{Pollock\index{Pollock, Sheldon} (1989:608)}
\end{myquote}

अतः अभी वेद की अपौरुषेयता की सिद्घान्त को स्पष्टतया वर्णन करेंगे~। प्रथमतः भगवान् जैमिनि ‘उक्तं तु शब्दपूर्वत्वम्' इस सूत्र पर विचार करेंगे~।

‘उक्तं तु शब्दपूर्वत्वम्' – यहॉ पर ‘शब्द' – शब्द से शब्द जन्य अध्ययन विवक्षित है~। तथा च सूत्रार्थ यह हुआ कि सभी पुरुषों का अध्ययन अध्ययनान्तर पूर्वक हुआ करता है, यह बात औत्पत्तिक सूत्र के आरम्भ में कह आये है~। वेद के अध्ययन करने वालों की शब्दपूर्वता अर्थात् अविच्छिन्न परम्परा है~। प्रत्येक वेदाध्येता के अध्ययन से पूर्व अन्य अध्येता का अध्ययन था~। इस रीति से शब्दाध्ययन की अविच्छिन्न परम्परा है~। निष्कर्ष यह है कि सभी लोग अपने गुरू ने जिस प्रकार अध्ययन किया उसी प्रकार अध्ययन करना चाहते है वेद का स्वतन्त्रतापूर्वक अध्ययन करने वाला कोई भी प्रथम अध्येता नही हुआ हैं~। जिसे उसका कर्ता कहा जा सके~। आज प्राचीन से भी प्राचीनतम वेदाध्येता का ज्ञान हमें है किन्तु उसके कर्ता का ज्ञान नहीं है~। अतः कर्तृस्मरण के अभाव में वेदों को अपौरुषेय ही समझना चाहिए~।

काठक, कालापकम् इत्यादि समाख्या के आधार पर वेदों को कठादिकर्तृक आक्षेप उठने पर यह समाधान किया जाता है– इस प्रकार कर्ता की कल्पना करना ठीक नहीं है क्योंकि कभी कभी लोग किसी ग्रन्थ के साथ किसी विशेषता को देखकर कर्तृभिन्न किसी अन्य व्यक्ति का नाम भी जोड दिया करते है~। उदाहरणार्थ– ‘सिद्धान्तकौमुदी' ग्रन्थ के कर्ता श्री भट्टोजी दीक्षित है, तथापि काशी के श्री देवनारायण तिवारी जी ने सिद्धान्त कौमुदी को ऐसी अद्भुत विशिष्ट शैली से आजीवन पढाया, जिस कारण लोग कौमुदी के साथ तिवारी जी का नाम जोड़कर ‘तिवारी जी की कौमुदी' कहने लग गये~। वह अद्भुत विशिष्ट शैली श्री तिवारी जी की अपनी ही थी~। वैसी शैली भारत वर्ष में अन्य किसी भी वैयाकरण की नही थी~। अतः यह सम्भव है कि जिस ग्रन्थ का कर्ता न भी हो, तथापि उसकी किसी विशेषता के कारण उसका नाम उस ग्रन्थ के साथ जोड़कर लोग उसके नाम पर भी उस ग्रन्थ को कह सकते है~। इस सम्भावना के आधार पर यहा भी कह सकते है कि कठादि महर्षियो ने भी अपनी निजी उत्कृृष्टतम शैली अपनी वेद शाखा का अध्यापन अवश्य किया होगा ~। कभी कभी इस प्रकार की उत्कृृष्टतम शैली से अध्यापन करने वाले भी होते है~। जैसे कि उदाहरण के रूप में ऊपर निर्देश कर चुके है~। 

इस प्रकार की अनन्य साधारणता अनेक प्रकार से प्राप्त हो सकती है~। ग्रन्थकर्ता के रूप में ग्रन्थ के व्याख्याकार के रूप में, ग्रन्थ के अध्यापन कर्ता के रूप में इत्यादि अनेक प्रकार है~। निष्कर्ष यह है कि मूलग्रन्थकार ही अनन्य साधारण नहीं होगा, ग्रन्थ व्याख्याकार भी, ग्रन्थ का अध्यापक भी अनन्यसाधारण समझा जा सकता है इस कारण वेदशाखाओं के साथ काठक आदि जुड़े हुए विशेषण सप्रयोजन है ~। वेदाध्यायी सभी लोगो को अच्छी तरह से स्मरण है कि वैशम्पायन महर्षि ने यजुर्वेद की समस्त शाखा का अध्ययन–अध्यापन किया था~। अतः बहुशाखाध्यायी पुरुषों की अपेक्षा एक – एक शाखा का अध्ययन–अध्यापन करने वाले कठादि महर्षियों के प्रवचन (अध्यापन) में असाधारणता रहना स्वाभाविक है~। अनेक शाखाओं के अध्यापन करने वालों के समक्ष केवल एक ही शाखा का अध्ययन करने वाला यह कठ महर्षि था~। इन्होने दूसरी शाखा का अध्ययन नहीं किया ~। अतः अपनी उस विशिष्ट शाखा में प्रकृष्टता प्राप्ति करने के कारण उसकी शाखा के साथ ‘काठक' यह जो असाधारण विशेषण जोड़ा गया है, वह उचित ही है~। 

प्रावाहणि, बवर इत्यादि के माध्यम से वेदों का अनित्यदर्शन जो कारण बताया तदर्थ–“जनन मरणवन्तो वेदार्थाः श्रूयन्ते” इस प्रकार की आक्षेप की प्राप्ति होने पर उसके समाधानार्थ ऐसा कहेंगे– किसी भी पुरुष का नाम ‘प्रवाहण' हो ऐसा आज तक श्रुत नहीं है~। जब ‘प्रवाहण' नामधारी ही कोई नही है तब उसका अपत्य प्रावाहणि बताना कैसे संगत हो सकता है? प्रावाहणि, बवर आदि शब्दो से प्रतीयमान यौगिक अर्थ की पर्यालोचना करने पर ‘प्रवाहणि' शब्द ‘प्रकर्षेण वाहयति' अर्थ को बताता है अर्थात प्रकर्षेण वहन क्रिया कर्तृक है~। उसी तरह बवर शब्द नित्य सिद्ध वायु आदि का वाचक है~। एवं च ‘प्र' शब्द प्रकर्ष के अर्थ में प्रसिद्ध है और वह धातु प्रापण पहुचाना, ले जाना के अर्थ में है~। किन्तु इन दोनों शब्दो का समुदायरूप ‘प्रावाहण' कही प्रसिद्ध नहीं है~। ‘प्रावाहणि' में इञ् प्रत्यय, जैसे अपत्य अर्थ में सिद्ध है, वैसे ही वह कर्तृविशिष्ट क्रिया में भी सिद्ध है~। अतः ‘प्रावाहणि' का अर्थ यह हुआ कि जो उत्कृष्ट रीति से वस्तु को ले जाए~। ‘प्रावाहण का पुत्र' (अपत्य) यह अर्थ नहीं~। उसी तरह प्रवहमान वायु की ध्वनि का अनुकरणमात्र ‘बवर' शब्द है~। वायु के शब्द की अनुकृतिरूप यह बवर शब्द, नित्यार्थ का अभिधायक होने से तद्घटित वाक्य “बवरः प्रावाहणिरकामयत” नित्यार्थ के ही अभिधायक (वाचक) सिद्ध हो रहे है (एवं च ये दोनों शब्द (प्रावाहणि और बवर) नित्य अर्थ को ही बता रहे है~। अनित्य अर्थ को नहीं~। इसलिये सूत्रकार ने कहा कि ये शब्द केवल श्रुतिसामान्य मात्र है, अर्थात् उनसे केवल अव्यक्त ध्वनि की समानता का बोध होता है~।

“वनस्पतयः सत्रमासत ‘सर्पाः सत्रमासत', गावो वा एतत् सत्रमासत” इत्यादि उन्मत्त बालप्रलाप सदृश सुनाई देने वाले असंगत वाक्य की प्राप्ति होने पर उसके समाधानार्थ जैमिनि मुनि कहते है ‘कृते वा विनियोगः स्यात् कर्मणः सम्बन्धात्' ‘कृते' कर्म में ‘विनियोगः' स्तुति के द्वारा उपयोग हो सकता है ‘कर्मणः' कर्मप्रतिपादक वाक्य ‘सम्बन्धात्' परस्पर साकांक्ष पदघटित होने से~। तात्पर्य यह है कि कर्मप्रतिपादक वाक्य (विधिवाक्य) परस्पर साकांक्ष पद घटित होने से ‘गावो वा एतत्सत्रमासत' इत्यादि वाक्यों का कर्म में स्तुति द्वारा उपयोग होता है~। निष्कर्ष यह है कि ‘गावो वा इत्यादि वाक्यों का स्वार्थ में तात्पर्य नहीं है~। किन्तु गो आदि जड़ पशुओ ने भी जब कर्मानुष्ठान किया तो विद्वान् लोग कर्म का अनुष्ठान करें इसमें सन्देह ही क्या है? इस प्रकार उन वाक्यों का कर्म की प्रशंसा करने में ही तात्पर्य है~। अतः सभी वाक्य ठीक है~।

“ज्योतिष्टोमेन स्वर्गकामो यजेत\index{yajna@\textit{yajña}}”, “सोमेन यजेत” इत्यादि वेदवाक्य परस्पर सम्बद्ध अर्थ के ही दिखाई देते है~। क्योंकि वे साध्य–साधन इतिकर्तव्यता विशिष्ट अर्थभावना विषयक विधि यज्ञनिषेध के ही प्रतिपादक है~। अर्थात् ‘वेदवाक्यानि परस्पर–सम्बद्धार्थ–पराणि साध्य साधनेति–कर्तव्यता–विशिष्टार्थ भावना–विषयकविधिनिषेधप्रतिपादकत्वात्' अतः इन वाक्यों को उन्मत्त बाल वाक्यों के तुल्य नही कहा जा सकता~। ज्योतिष्टोमादि सभी वाक्य क्रियापरक है अतः परस्पर सम्बद्धार्थक ही दिखाई दे रहे है~। क्रिया को ही ‘भावना' शब्द से भी कहा जाता है~। प्रत्येक क्रिया के साथ (उद्देश्य) साधन (उपाय करण) और इतिकर्तव्यता (क्रिया करने की पद्धति अर्थात् कर्तव्य विशेष) ये तीन अंश होते है, इसलिये वह (क्रिया) कभी निरर्थक नही हुआ करती~।

अन्यत् ‘वनस्पतयः' इत्यादि वाक्य भी असंगत (असम्बद्धार्थक) नहीं है~। क्योंकि आगे कहे जाने वाले सत्रयाग को स्तुति की अपेक्षा (आवश्यकता) है, तब ये वाक्य, उसकी अपेक्षित स्तुति का समर्पण करके सार्थक हो जाते है, और उसका प्रामाण्य सिद्ध हो जाता है~। अभिप्राय यह है कि ये वाक्य साक्षात् क्रियाप्रवर्तक नहीं है किन्तु सत्रयाग के (स्तावक) अर्थवाद है~। सत्रयाग की स्तुति करने का प्रकार ‘वनस्पतयो नाम अचेतना' है जबकि अचेतन (जड़) वनस्पतियों ने भी इस सत्र (याग) का अनुष्ठान किया तब विद्वान् ब्राह्मण यज्ञानुष्ठान\index{yajna@\textit{yajña}} करें, क्या इसे कहने की आवश्यकता होगी? जैसे लोकव्यवहार में कहा करते है कि सायंकाल के समय मृग (पशु) भी नहीं चरते, तब विद्वान् ब्राह्मणों के विषय में तो कहना ही क्या होगा? अर्थात् कर्तव्य–अकर्तव्य के विवेक से रहित रहने वाले पशु भी सायंकाल के समय अपने स्वच्छन्द विहार का त्याग कर स्वस्थ होकर चुप–चाप खड़े रहते है, इसलिये विद्वान् ब्राह्मण (विचारशील ब्राह्मण) को भी सायंकाल के समय स्वस्थ शान्तचित्त होकर परमेश्वर की आराधना करना चाहिए, क्या यह कहने की आवश्यकता होगी? अर्थात् नहीं~। किञ्च वेदों का जो उपदेश हैं, वह वेदाध्ययन करने वाले (अनिन्दित) शिष्टों की परम्परा से जाना जाता है, तथा सन्मित्र के उपदेश (परामर्श) के समान होने से सर्वथा अनाशङ्कित है अर्थात् दोष का लवलेश भी उसमें नहीं है~। अतः उसे उन्मत्त बालवाक्यसदृश कहनें की धृष्टता कैसे की जा सकती है~। एवञ्च उनके प्रति किसी प्रकार भी दुष्ट आशंका नही करना चाहिए~। तस्मात् चोदनावाक्यों\index{codana@\textit{codanā}} के अपौरुषेय होने से उनका प्रामाण्य सिद्ध हो जाता है~।

प्रकारान्तर से भी हम वेद की अपौरुषेयता को सिद्ध कर रहे है~। प्रथमतः शब्द क्या है तथा अर्थर् क्या है, शब्द और अर्थ दोनों का परस्पर सम्बन्ध क्या है, इस पर हमें विचार करना चाहिए~। सम्बन्ध दो सम्बन्धियों को अपना आधार बनाकर रहता है~। जब तक उन आधारभूत सम्बन्धियों को अर्थात् उनके स्वरूप को न बताया जाय तब तक सम्बन्ध का निरूपण (सम्बन्ध के बारे में कुछ कहना) करना सम्भव नही है~। इसलिये सम्बन्ध के आधारभूत शब्द का स्वरूप क्या हैं? उसीं का प्रथमतः विचार कर लें – इस अभिप्राय से भाष्यकार ‘अथ गौरित्यत्र' भाष्य का आरम्भ कर रहे हैं~। शब्द के स्वरूप का निर्धारण करने के पश्चात् शब्दार्थ सम्बन्ध की नित्यता को सिद्ध करेंगे~।

\vskip -1cm


\section*{शब्दस्वरूपविचारः}

प्रथमतः यह विचार ले कि ‘गौः' ऐसा उच्चारण करने पर हम ‘शब्द' के रूप में किसे जानते है? भगवान् उपवर्ष तो ‘गौः' में गकार औकार, विसर्जनीय को शब्द कहते है~। क्योंकि श्रोत्र (कर्ण) से ग्रहण किये जाने के अर्थ में ‘शब्द' शब्द का व्यवहार लोक में प्रसिद्ध है~। वे गकार–औकार और विसर्ग श्रोत्र से ग्रहण किये जाते है~। पूर्व–पूर्व वर्ण के सुनने पर उनसे एक–एक संस्कार उत्पन्न होता जाता है~। उन उत्पन्न हुए संस्कारो के सहित जो अन्तिम वर्ण रहेगा, वही वर्ण ‘अर्थ' का बोधक होता है~। निष्कर्ष यह है कि ‘ग' वर्ण के सुनने पर उससे संस्कार उत्पन्न होगा, उस संस्कार के सहित जो ‘औ' वर्ण का श्रवण होगा, उससे भी एक संस्कार उत्पन्न होगा, उन दोनों संस्कारो के सहित जो अन्तिम वर्ण विसर्ग का श्रवण होता हैं उससे अर्थ का बोध होता है~। इस रीति से शब्द को अर्थप्रत्यायक कहने में कोई दोष नहीं है~। अर्थ यह है कि ‘गौः' इस प्रकार उच्चारण करने पर गोत्वरूप अर्थ का प्रत्यायक कहा जाने वाला जो शब्द है, वह क्या गकारादि वर्णरूप है? इस प्रकार शब्द का स्वरूप क्या है, यह पूछकर वृद्धसम्मति प्रदर्शित करते हुए अपना स्वयं का मत (गकारादि वर्ण ही शब्द का स्वरूप है) ऐसा बतायेंगे~। शब्द का वर्णरूप होना जो उपवर्ष के नाम पर बताया गया है वह स्वमत की पुष्टि के लिये वृद्धसम्मति के रूप में बताया गया है~। प्रत्यक्ष और अर्थप्रत्यायकत्व के आधार पर मीमांसको ने यह सिद्ध किया है कि शब्द का स्वरूप गकारादि वर्ण ही है~। पहले वह प्रत्यक्ष के आधार पर वर्णो की शब्द स्वरूपता को बताने के लिये किया गया है कि “श्रोत्रग्रहणे हि अर्थे लोके शब्द शक्ति प्रसिद्धः~। ते च श्रोत्रग्रहणाः” भाष्य उपस्थित किया गया है~। इस भाष्य से यह बताया गया हैै कि श्रोत्रेन्द्रियजन्य प्रत्यक्ष का विषय होने वाले (अर्थ) में ‘शब्द' इस शब्द की प्रसिद्धि सर्वत्र है अर्थात् लोग उसे (शब्द को) वाचक कहते है~। वे गकारादि वर्ण ही श्रोत्र से ग्राह्य होते है~। एवं च लोकव्यवहार में ‘श्रोत्रग्राह्यत्वं शब्दत्वम्' इस लक्षण से लक्षित को ही शब्द नाम से कहा जाता है~। अतः गकारादि वर्ग ही शब्द का स्वरूप है, यह समझना चाहिए~। 

अक्षरों में अर्थवान् के प्रति जो निमित्तता निमित्तभाव है, उसे गौण नही कह सकते, क्योंकि ‘तद्भावे भावात् तदभावे चाभावात्' अक्षरों के होने पर अर्थप्रतीति होती है और अक्षरों के न होने पर अर्थप्रतीति नही होती है~। अतः अर्थप्रतीति करानें में अक्षरों को गौणरूप से निमित्त नहीं कहा जा सकता~। गो शब्द में गकार आदि के अतिरिक्त अन्य किसी गो शब्द का प्रत्यक्ष नहीं हो रहा है, अर्थात् ग, औ तथा विसर्ग के सिवाय अन्य कोई गो शब्द से प्रत्यक्ष नही है~। अतः भेदज्ञान के न होने से अभेदज्ञान हो रहा है~। गकारादि अर्थात् ग औ तथा विसर्ग तो अक्षर है, वे ही पद (शब्द) है~। अर्थात् वर्णों की ही ‘शब्द' यह संज्ञा है~। अतः उनसे (गकार, औकार, विसर्जनीय) (इन अक्षरों से) भिन्न अन्य कोई नहीं है~। जिसे पद (शब्द) कहा जा सके ~। 

\newpage

तथा च अन्वय व्यतिरेक के देखने से यह अवगत होता है कि अर्थप्रतिपत्ति के होने में अक्षर ही निमित्त है~। एवं च अर्थप्रतिप्रत्ति में अक्षर स्वव्यापार के द्वारा हेतु है~। उससे भिन्न अन्य किसी प्रकार का हेतुत्व कहीं पर भी दिखाई नहीं देता और जो हेतु होता है उसके व्यापार का व्यवधान तो सर्वत्र ही नियत रहता है~। संस्कार तो शब्द का व्यापार हीं है~। अतः उसका जो व्यवधान है, वह अव्यवधान ही है~। इसलिये अक्षरों में ‘शब्द' शब्द का प्रयोग गौण नहीं हो एवं च वर्णो में ‘शब्द' शब्द का मुख्य प्रयोग है गौण नही है, अर्थात् वर्ण ही शब्द है~।


\section*{शब्द के अर्थ का विचार}

“अथ गौरित्यस्य शब्दस्य कोऽर्थः? सास्नादिविशिष्टाकृतिं ब्रूमः” गौः इत्यादि प्रश्नभाष्य का आशय यह है कि गोत्व इत्यादि सामान्य–रूप आकृति का निरूपण करना बड़ा कठिन होगा~। अतः आकृति नाम की कोई वस्तु प्रसिद्ध न होने कारण अगोव्यावृत्तिरूप ‘अपोह' को यदि शब्दार्थ कहें तो अपोह नाम की कोई वस्तु ही नहीं है, तब उसको आधार मानकर उनका (शब्द–अर्थ का) सम्बन्ध नित्य कैसे हो सकेगा?

अतः प्रत्यक्ष\index{pratyaksa@\textit{pratyakṣa}}–प्रमाण\index{pramana@\textit{pramāṇa}} से ‘अयमपि गौः ‘शाबेलयादन्यो–बाहुलेयः' शाबेलय नामक गौः से बाहुलेय नामक गौः भिन्न है, इस प्रकार की प्रतीति होने के कारण भिन्न–अभिन्न रूप से प्रतीत होने वाला गोत्व आदि सामान्य सिद्ध हो जाता है~। इसलिये वार्तिककार ने कहा है–

\vskip 4pt

\begin{verse}
“प्रत्यक्ष–बल–सिद्धस्य सामान्यस्य कुतर्कतः~।\\न शक्योऽन्हवः कर्तुं सर्वं विजयते हि तत्~।।”
\end{verse}

\vskip 4pt

इत्यादि रीति से उसका (गोत्वादि सामान्य का) अपोह (छिपाना) करना शक्य नहीं है~। यतः गोत्वादिरूप सामान्य ही शब्दार्थ है, इस कारण उसका (गोत्वादिरूप सामान्य का) आधारभूत सम्बन्ध भी नित्य है, यह सिद्ध हो ही जाता है, इस सिद्धान्त को निश्चित करके सिद्धान्तभाष्य– “सास्नादिविशिष्टाकृति” कहा गया है~। “जातिमेवाऽऽकृतिम्प्राहुर्व्यक्तिराक्रियते यथा” वार्तिककार के इस वचन से जाति को ही आकृति शब्द से कहा जाता है~। आक्रियते का अर्थ है – निरूप्यते~। इसी अभिप्राय से भाष्यकार ने “सास्नादिविशिष्टाकृतिः इति ब्रूमः” कहा है, अर्थात् एक गोपिण्ड (गोव्यक्ति) में समुच्चित रूप से रहने वाले सत्तादिरूप अनेक सामान्यों के मध्य में से गोशब्दवाच्य सामान्य का अलग से निर्देश करने के लिये अर्थात् उसके साथ (गोत्वरूप–सामान्य के साथ) एक ही अवयवी व्यक्ति में असाधारण रूप से विद्यमान रहने वाले उपलक्षणभूत सास्नादि अवयवों के द्वारा अन्य अवयवों से अलग निर्देश करने के लिए सास्नादिविशिष्टाकृति कहा जाता है~।

\newpage

\section*{शब्द और अर्थ दोनों का परस्पर सम्बन्ध}

शब्द और अर्थ दोनों में जो सम्बन्ध है, वह पुरुषनिर्मित नहीं है, अर्थात् अपौरुषेय है~। अतः धर्माऽधर्म\index{dharma@\textit{dharma}} के बोधन में वेद का प्रमाण\index{pramana@\textit{pramāṇa}} सिद्ध है~। शब्दार्थ के परस्पर सम्बंध जोड़ने वाले पुरुष का अभाव होने से हम जानते है कि शब्द और अर्थ का परस्पर सम्बन्ध पुरुषनिर्मित नहीं है यानी अपौरुषेय है~।

यदि कोई पुरुष शब्दार्थ सम्बन्ध का निर्माता हुआ रहता तो अवश्य ही किसी न किसी को उसका प्रत्यक्ष हुआ होता, किन्तु किसी को भी उसका प्रत्यक्ष कभी भी नहीं हुआ है~। अतः प्रत्यक्ष प्रमाण के अभाव से हमने यह जाना कि शब्दार्थ को जोड़ने वाला कोई पुरुष नहीं है~। अनुमानादि अन्य प्रमाणों से भी शब्दार्थ सम्बन्ध जोड़ने वाले व्यक्ति का ज्ञान नहीं हो सकता क्योंकि अनुमानादि अन्य प्रमाण\index{pramana@\textit{pramāṇa}} भी प्रत्यक्ष प्रमाणपूर्वक ही हुआ करते है~। जबकि जोड़ने वाले व्यक्ति का ज्ञान प्रत्यक्ष प्रमाण से ही पता नहीं चल पाया तो प्रत्यक्ष पूर्वक प्रवृत्त होने वाले अनुमानादि अन्य प्रमाणों से उसका ज्ञान कैसे हो पायेगा~।

यदि यह कहा जाय कि बहुत समय बीत जाने के कारण जैसे उसका प्रत्यक्ष नही हो रहा हैं, वैसे ही दीर्घतर काल बीतने से उसे स्मरण भी नही हो पा रहा है~। चिरवृत्त होने से उसका स्मरण न हो, यह भी नही कह सकते~। चिरवृत्त हुए युग के युग बीत गये, किन्तु आज तक अविच्छिन्न रूप से राम, बुद्ध\index{Buddha, the}, कुमारिल\index{Kumarila@Kumārila}, प्रभाकर\index{Prabhakara@Prabhākara}, शंकराचार्य\index{Sankaracarya@Śaṅkarācārya} प्रभृति लोगों का स्मरण सभी को है~। शब्दार्थ व्यवहार अविच्छिन्न (अटूट) परम्परा से चला आ रहा है~। यहा पुरुषों का शब्दार्थ व्यवहार का अभाव नहीं हुआ है~। इसलिये शब्दार्थ सम्बन्ध करने वाले पुरुष का विस्मरण होने का कोई कारण ही नही है~। महाकवि कालिदास, भारवि, भवभूति की तरह शब्दार्थ सम्बन्ध निर्माता का स्मरण अविच्छिन्न परम्परा के कारण अवश्य ही रहना चाहिए, किन्तु किसी को भी वह आज तक स्मरण नहीं है~। इसलिये कहा जा सकता है कि शब्दार्थ के सम्बन्ध का निर्माता कोई भी नही था~। वह सम्बन्ध अपौरुषेय है और अपौरुषेय होने से वह नित्य है~। 

यदि कोईं पुरुष किसी शब्द का अर्थ से सम्बन्ध जोड़कर अन्य लोगों से उस शब्दार्थ का व्यवहार चलाया होता तो प्रत्येक व्यक्ति को व्यवहार करते समय शब्दार्थ सम्बन्ध करने वाले उस व्यक्ति का स्मरण अवश्य ही हुआ होता~। सम्बन्धकर्ता और व्यवहारकर्ता दोनों का ऐक्य अर्थात् समान ज्ञान होने पर ही दोनों का प्रयोजन सिद्ध हुआ करता है~। विरुद्ध ज्ञान यदि दोनों का रहें तो व्यावहारिक प्रयोजन सिद्ध नहीं हो पाता~। जैसे–उदाहरणार्थ व्याकरणसूत्रकार पाणिनि मुनि के व्यवहार को न जानने वाले लोगों को पाणिनिकृत पारिभाषिक ‘वृद्धि' शब्द से ‘आदैच्' अर्थात् आ, ऐ, औ का ज्ञान नहीं हो पाता अथवा पाणिनि के मत को न मानने वालें को ‘वृद्धि' शब्द से आ, ऐ, औ की प्रतीति नहीं होती~। वृद्धि शब्द से आदैच् का ज्ञान उन्हीं को हो पाता है जिन्हें सम्बन्धकर्ता पाणिनि का स्मरण है क्योंकि पाणिनि ने ही ‘वृद्धि' शब्द और ‘आदैच्' में परस्पर सम्बन्ध स्थापित किया है~। उसी तरह का दूसरा उदाहरण ‘मगण' कहने पर छन्दः सूत्रकार पिङ्गल के व्यवहार को न जानने वाले लोग उस त्रिक को समझ नहीं पाते जिसमें तीनों अक्षर गुरु हुआ करते है~। अथवा पिङ्गल की कृति– अर्थात् ‘मगण' और सर्वलघु त्रिक के सम्बन्ध को न मानने वाले लोगों को ‘म' कहने से सर्वगुरुत्रिकरूप की प्रतीति नहीं हो पाती क्योंकि उन्हें या तो ‘म' शब्द और त्रिकरूप अर्थ के सम्बन्धरूकर्ता का स्मरण नही है, या उस छन्दः सूत्रकार पिंगल के सिद्धान्त को ही वे नही मानते~। मगण कहते ही सर्वगुरुत्रिक का उन्हें ही स्मरण होता है जिन्हें उन दोनों का समान ज्ञान हो जाता है दोनों का समान ज्ञान आवश्यक है~। अर्थात् सम्बन्धकर्ता और व्यवहर्ता दोनों का ऐक्य होना नितान्त आवश्यक है~। उपर्युक्त विवेचन से यह स्पष्ट हो रहा है कि वैदिक व्यवहार करने वाले अर्थात् वेदार्थ वक्ता भी शब्दार्थ सम्बन्ध और व्यवहार के निर्माता का स्मरण अवश्य ही करते~। ‘वृद्धिरादैच्' सूत्र के कर्ता का विस्मरण होने पर ‘वृद्धिर्यस्याचामादिस्तद् वृद्धम्' सूत्रगत ‘वृद्धि' शब्द से कुछ भी ज्ञान नहीं हो पायेगा अर्थात् सूत्र की अर्थ को समझ ही नहीं सकेंगे~। इस विवेचन से यह समझ में आता है कि किन्ही पुरुष ने शब्दों का अर्थ के साथ सम्बन्ध निर्माण करके उनका व्यवहार कराने के लिये वेदों की रचना की हो, यह सम्भव नही है~। यद्यपि शब्दार्थ सम्बन्ध के निर्माता का विस्मरण सम्भव हो सकता है, तथापि किसी प्रबल प्रमाण\index{pramana@\textit{pramāṇa}} के अभाव में सम्बन्धकर्ता की कल्पना नहीं की जा सकती~। यह बात सत्य है कि कभी–किसी के विद्यमान रहनें पर भी उसका प्रत्यक्ष नही हो पाता तथापि उसको आधार मानकर हम बिना किसी प्रबल प्रमाण के शशविषाण की सत्ता को स्वीकार नहीं कर सकते अर्थात विद्यमान वस्तु की अप्रत्यक्षता के आधार पर अविद्यमान को भी विद्यमान नहीं बनाया जा सकता क्योंकि जिस प्रकार अप्रत्यक्षत्व और विद्यमानत्व में कोई व्याप्तिसम्बन्ध नहीं है, उसी प्रकार विस्मृतत्व और सम्बन्धों के अस्तित्व में भी व्याप्ति–सम्बन्ध नहीं है~। अतः शब्द और अर्थ का सम्बन्ध अपौरुषेय है~।

शब्द का उपदेश (कथन) सिद्ध अर्थात् पहले से ही विद्यमान पदार्थ के समान होता है~। उनमें अर्थबोध करानें की शक्ति पहले से ही निहित रहती है~। इसी अभिप्राय को बताने के लिये भाष्यकार ‘सिद्धवत् उपदेशात्' कह रहे है~। इस कथन से यह स्पष्ट हो जाता है कि सूत्र का ‘उपदेश' शब्द सम्बन्ध–करण को नहीं बता रहा है, अपितु प्रसिद्ध सम्बन्ध का ही उपदेश कर रहा है~। एवं च शब्दार्थ–सम्बन्ध का कथन ही किया जाता है, उसे (शब्दार्थ–सम्बन्ध को) जोड़ा नही जाता~। यदि यह वस्तुस्थिति रहती की शब्दार्थ सम्बन्ध के निर्माता को स्वीकार न करने पर नियमतः अर्थ का ज्ञान नहीं होता, तो शब्दार्थ सम्बन्ध के निर्माण को अर्थापत्ति\index{arthapatti@\textit{arthāpatti}} प्रमाण\index{pramana@\textit{pramāṇa}} से मान भी लिया जाता, किन्तु शब्द से अर्थ की प्रतीति होने का अन्य उपाय भी है~। वह उपाय है–व्यवहार~। अतः सम्बन्ध को स्वाभाविक (अपौरुषेय) स्वीकार करने पर भी अर्थ का ज्ञान व्यवहार से भी हो जाता है~। उसी को उदाहरण के द्वारा सिद्धान्ती के मुख से भाष्यकार बता रहे है–अपने किसी प्रयोजन के उद्देश्य से शब्द प्रयोग रूप व्यवहार करने वाले वृद्ध के (द्वारा कहे जाने वाले) शब्दों के अर्थ को समझते दिखाई देते है~। ये शब्दप्रयोक्ता वृद्ध भी जब स्वयं बालक थे तब उन्होनें अन्य वृद्धों से उन्होनें भी अन्य वृद्धों से – इस परम्परा का कोई आदिकाल नहीं है~। इस प्रकार की ‘अनादि–वृद्ध–व्यवहार–परम्परा' से शब्दार्थ सम्बन्ध के ज्ञान होता आया है~। शब्दार्थ सम्बन्ध के ज्ञान का उपाय जबकि वृद्धव्यवहार है, तो उसके रहते हुए अर्थापत्ति प्रमाण के सहारे सम्बन्ध निर्माता की कल्पना नहीं की जानी चाहिए~। हम देखते है कि किसी शब्द का किसी अर्थ के साथ सम्बन्ध जोड़ने के लिये कुछ शब्दों के अर्थों का ज्ञान अवश्य अपेक्षित रहता है~। उदाहरणार्थ उत्पन्न हुए शिशु का नामकरण करने के लिये कतिपय सार्थक नामों का ज्ञान रखना पड़ता है~। यदि आरम्भकाल में किसी शब्द का किसी अर्थ के साथ कोई किसी प्रकार का भी सम्बन्ध नहीं था तो उसे जोड़ा कैसे गया? अतः यही स्वीकार करना होगा कि वृद्धव्यवहार से ही शब्दार्थ सम्बन्ध को जाना जाता है~। इस प्रकार अनादि–वृद्ध–व्यवहार से ही अर्थ–प्रतीति की अनायास उपपत्ति हो जाती है तब अनुपपत्ति के अभाव में उसकी महिमा के सहारे सम्बन्धकर्ता की कल्पना नहीं कर सकते~। कोई देश ऐसा नहीं है जो शब्दार्थ सम्बन्ध से रहित हो~। जिस प्रकार इस देश में सास्ना (गल–कम्बल) वाले पशु में गो शब्द का प्रयोग किया जाता है, उसी प्रकार समस्त दुर्गम स्थलों में भी सास्नावाले पशु में ही गो शब्द का प्रयोग होता है~। शब्दार्थ सम्बन्ध के निर्माता अनेक लोगों का परस्पर सम्मिलन दुर्गम स्थानों में कैसे सम्भव हो सकता है? यदि किसी एक को ही सम्बन्धकर्ता कहें तो उस एक व्यक्ति के लिए सब जगह जा–जा कर सार्वजनीन व्यवहार की शिक्षा देना सम्भव नहीं है~। अतः न तो अनेक सम्बन्धकर्ता है और न एक ही सम्बन्धकर्ता है~। एवञ्च शब्दार्थ के सम्बन्ध का कर्ता कोई भी नही है~। ‘अव्यतिरेकश्च' पद की व्याख्या इस प्रकार भी करते है–शब्दार्थ सम्बन्ध से रहित कोई काल नहीं है~। जिसमें कोई भी शब्द किसी भी अर्थ से सम्बद्ध नहीं था अर्थात् सभी भूत, भविष्यत् एवं वर्तमान तीनों कालों में शब्दार्थ सम्बन्ध सदा विद्यमान है~। यह शब्दार्थ सम्बन्ध तो अनादिकाल से चला आ रहा है~।

अतः शब्द का अर्थ के साथ अपौरुषेय सम्बन्ध है~। इसी कारण “तत् प्रमाणम्\index{pramana@\textit{pramāṇa}} अनपेक्षत्वात्” – यह भाष्य बता रहा है कि वह शब्द निरपेक्ष होने से प्रमाण है~। इस विवेचन से यह स्पष्ट है कि वैदिक शब्द न तो किसी ऐसे अन्य पुरुष की अपेक्षा रखता है, जो शब्द को प्रमाणित करें और न ही उसे समर्थित करने के लिये किसी अन्य ज्ञान की अपेक्षा है~। इसलिये “चोदनालक्षण\index{codana@\textit{codanā}} एव धर्मो, नान्यलक्षणः”~। अतः निष्कर्ष यह हुआ कि वेद से ही धर्म\index{dharma@\textit{dharma}} का ज्ञान होता है, किसी अन्य प्रमाण से उसका ज्ञान नहीं हो सकता~।

\newpage

\section*{बुद्ध\index{Buddha, the} तथा जैमिनि\index{Jaimini} के काल का विचार}

शेल्डन पॉलॉक\index{Pollock, Sheldon} अपनी पुस्तक \enginline{‘Language of the Gods in the World of Men'} में यह प्रतिपादित किये है कि वेद की अपौरुषेयता मूलतः वैदिक संस्कृति से नहीं जुडा है, जैमिनि मुनि के द्वारा इस सिद्धान्त को प्रथमतः स्थापित किया गया~। वेदों पर किसी भी प्रकार का प्रश्न न उठ सकें इसलिये जैमिनि मुनि ने वेद को अपौरुषेय बताया तथा उसकी प्रामाण्यता को प्रश्नातीत दर्शाया~।

\begin{myquote}
\enginline{“...It also seems likely that \textit{atleast some of the most salient articulations of the world, what we now tend to think of as its foundational principles, may have first been conceptualized as a defensive, even anti–axial, reaction to Buddhism...} It is self–evident that no one would elaborate propositions of the sort we find Mimamsakas\index{Mimamsaka@Mīmāṁsaka} to have elaborated, such as the thesis of the authorlessness of the Veda, unless the authority of the veda and its putative authors had first been seriously challenged” (Italics mine)}\hfill \enginline{Pollock (2005:397)}
\end{myquote}

\vskip 4pt

\begin{myquote}
\enginline{“The explicit formulation of what are now rightly viewed as axioms that naturalized the social world and the world of discourse–restrictions on the right to sacrifice and on the originary relationship of word and meaning (the \textit{adhikara} and \textit{autpattika} doctrines discussed earlier) as well as the notion of an authorless and eternal Veda existing entirely outside of history – were likely developed in response to the Buddhist critique: Neither make sense without the arguments to the contrary.”}

~\hfill \enginline{Pollock\index{Pollock, Sheldon} (2006:53)}
\end{myquote}

\vskip 4pt

\begin{myquote}
\enginline{“What was at stake for Mīmāṁsaka–s\index{Mimamsaka@Mīmāṁsaka} in asserting the uncreated, eternal nature of language was the possibility that Vaṅmaya or a thing–made–of–language that is, a text like the Veda – could be eternal too, something the Buddhists sought fundamentally to reject.”}\hfill \enginline{Pollock (2006:397)}
\end{myquote}


\section*{बुद्ध\index{Buddha, the} के पूर्ववर्ती जैमिनि\index{Jaimini} मुनि}

\vskip 3pt

जैमिनिसूत्रों की रचना का काल निर्णय करने में पाश्चात्य विद्वानों ने अपनी बुद्धि का अपव्यय ही किया है~। डा. कीथ (\enginline{Keith}) तथा डा. दास गुप्ता (\enginline{Das Gupta}) ने इन सूत्रों का रचनाकाल ईसा पूर्व \enginline{200} वर्ष बताया है~। डा. राधाकृष्णन् (\enginline{Radhakrishnan}) ने इन सूत्रों की रचना के काल की कल्पना ई.र्पू.\enginline{400} शताब्दी तक की है, इसके आगे नही बढ़ पाये है~। इसी प्रकार अनेक ऐतिहासिको ने अनिश्चित आधार पर भिन्न – भिन्न कल्पनाओं को जनता के सामने उपस्थित किया है~। उसका परिणाम यह हुआ कि सर्वसाधारण जनता के मस्तिष्क में भ्रम उत्पन्न हो गया, क्योंकि इन काल्पनिकों की कल्पनाओं में ऐक्य नहीं है~। अतएव मैक्डानल \enginline{Macdonell\index{Macdonell, Arthur A.}} नामक पाश्चात्य विद्वान् का कहना है कि भारतीय ऋषि–महर्षियों के अथवा उनकी रचनाओं के काल का निर्णय करना आकाशपुष्पों को तोड़ने के समान है~। इस प्रकार के काल्पनिक कालनिर्णय में मैक्डानल का किंचिन्मात्र भी विश्वास नहीं है~। इस तथ्य की जानकारी भारतीय विद्वानों को पहले से ही था~। अतएव भारतीय शिक्षा–दीक्षित विद्वानों में से किसी ने भी ऐसी निराधार अटकले बांधने में अपनी बुद्धि का अपव्यय नहीं किया है~। 

\vskip 2pt

जैमिनि\index{Jaimini} के नाम पर अनेक ग्रन्थ पाये जाते है~। जैसे–जैमिनीय शाखा, जैमिनीय ब्राह्मण, जैमिनि कोशसूत्र, जैमिनीय निघण्टु जैमिनिपुराण, जैमिनि भागवत, जैमिनिसूत्र, जैमिनि सूत्रकारिका, जैमिनिस्मृति, जैमिनीय श्रौतसूत्र, जैमिनिगृह्यसूत्र आदि~। अतः जिस जैमिनि ने मीमांसासूत्रों की रचना की है उसके काल का विचार जैमिनि के सूत्रों के आधार पर विचार करने की अत्यधिक आवश्यकता है~। 

\vskip 2pt

महाभाष्य में ‘मीमांसक' शब्द का उल्लेख बार–बार किया गया है, जिसे दर्शाते है ‘अथेह कस्मान्न भवति यात्रिकश्चायम् वैयाकरणश्च कठश्चायं बड्वृचश्च औक्थिकश्चायं मीमांसकश्च' (महाभाष्य\index{Mahabhasya@Mahābhāṣya} अ.\enginline{2} पा.\enginline{2} सू \enginline{29}) इन उल्लेखो से स्पष्ट होता है कि पतञ्जलि के समय मीमांसा का प्रचार पर्याप्त हो चुका था~। अतः सहज अनुमान होता है कि मीमांसा सम्प्रदाय के प्रथम सूत्रकार जैमिनि\index{Jaimini} का अस्तित्व भगवान् पतञ्जलि के पूर्व ही था~।

\vskip 2pt

‘अथ गौरित्यत्र कः शब्दः? गकार–औकारविसर्जनीया इति भगवान् उपवर्षः (शाबर भा.प्र.भाग) शबरस्वामी\index{Sabarasvamin@Śabarasvāmin} के इस लेख से अवगत होता है कि जैमिनीय मीमांसा सूत्रों के वृत्तिकार उपवर्ष थे~। अतः कथासरित्सागर के अनुसार पाणिनी के सूत्रों पर ‘वार्तिक' की रचना करने वाले कात्यायन के समकालिक ‘उपवर्ष' को मानना होगा~। तब व्याख्याकार की अपेक्षा मूल ग्रन्थाकार को पूर्ववर्ती कहना होगा~। इसलिये उपवर्ष के पूर्व ही जैमिनि को मानना होगा और उपवर्ष के समकालिक वार्तिककार से भी प्राचीन ‘जैमिनि' को कहना चाहिए~। इतना ही नही व्याकरण सूत्रकार पाणिनि भी क्रमादिगण में ‘मीमांसा' का पाठ कर स्वयं अपने को जैमिनि का पश्चाद्वर्ती होना बताया है~।

\vskip 2pt

तैत्तिरीय प्रातिशाख्य में मीमांसक' शब्द का उल्लेख प्राप्त होता है~। ‘मीमांसकानां च मीमांसकानां च' (तैत्तिरीय\index{Taittiriya Pratisakhya@\textit{Taittirīya Prātiśākhya}} प्राति.अ.\enginline{5}, सू.\enginline{41}) इस प्रातिशाख्य के तीन भाष्यों में से एक का कर्ता (वररुचि) को बताया जाता है~। ‘व्याख्यानं प्रातिशाख्यस्य वीक्ष्य वाररुचादिकम्”~। कृतं त्रिभाष्यरत्नं यद् भासते भूसुरप्रियम्'~। (तैत्ति\index{Taittiriya Pratisakhya@\textit{Taittirīya Prātiśākhya}} प्रा. त्रिभाष्यरत्नोपक्रम) यह वररुचि यदि वार्तिककार वररुचि है तो तैत्तिरीय प्रातिशाख्य की रचना के समय ही ‘मीमांसाशास्त्र' ने अत्यधिक प्रसिद्धि प्राप्ति कर ली तथा ‘पाणिनि' को वररुचि के समकालिक यदि मानते है तो पाणिनी, वार्तिककार और भाष्यकार तीनों की अपेक्षा मीमांसासूत्रकार महर्षि जैमिनि अत्यन्त प्राचीन सिद्ध होते है~।

\vskip 2pt

मीमांसासूत्रों को सूक्ष्म दृष्टि से परीक्षण करने पर यह ज्ञात होता है कि बुद्ध\index{Buddha, the} का उल्लेख कहीं पर भी प्राप्त नहीं होता है~। महामहोपाध्याय पी.वी.काणे \enginline{Kane\index{Kane, P. V.}} तथा डा.कीथ कहते है– \enginline{“There is absence of any express reference to Buddhist dogma and Philosophy” (Devasthali\index{Devasthali, G. V.} 1939: 65)} परन्तु मीमांसासूत्रों में बुद्ध का उल्लेख क्यूँ नही है? इसका समाधान करते हुए वो कहते है कि बुद्ध का उल्लेख करने के लिए मीमांसा शास्त्र में न उसकी आवश्यकता है न प्रसंग~। यह बात स्वीकार योग्य नही है~। बुद्ध ने आत्मा को क्षणिक बताते हुये कर्मकाण्ड को बलवत् रूप से खण्डन किया है, यदि जैमिनि\index{Jaimini} बुद्ध के पश्चाद्वर्ती है तो उन्होनें निश्चित रूप से बुद्ध की इन बातों को खण्डन किया होता, किंतु कहीं पर भी बुद्ध का प्रसंग नहीं उठाया है~। जहाँ जैमिनि ने बुद्ध शब्द का प्रयोग किया है वो सामान्य अर्थ में है~। ‘बुद्धशास्त्रात्' (\enginline{1.2.33}) इस सूत्र में बुद्ध शब्द का निर्वचन (कर्म को जिसने जाना हो) यह है~। यदि जैमिनि के समय में बुद्ध\index{Buddha, the} सुप्रसिद्ध थे तर्हि वो इस पद को सामान्य अर्थ में नहीं प्रयोग करते~। इससे यह सिद्ध होता है कि जैमिनि\index{Jaimini} बुद्ध के पूर्ववर्ती ही है~। जी.वी.देवस्थली कहते है– \enginline{“The absence of any reference to Buddhist doctrines would thus appear to be a clear proof of the \textit{Mimamsa sutra–}s\index{Mimamsasutra@\textit{Mīmāṁsā sūtra}} belonging to a date prior to the rise of Buddhism in India.”(Devasthali\index{Devasthali, G. V.} 1939: 65)}

\vskip 2pt

और तो और भगवान् उपवर्ष की बातों में बुद्ध के आक्षेपों का समाधान दिखाई देता हैं~। अतः उपवर्ष के समय में प्रायः बुद्ध की ख्याति हो चु की थी~। इससे स्पष्ट होता है कि जैमिनि के पश्चाद्वर्ती और उपवर्ष के पूर्ववर्ती बुद्ध थे~। 

\vskip 2pt


\section*{प्रकारान्तर से स्वमत की पुष्टि के लिये अन्यत् हेतु}

\vskip 2pt

ऐसा प्रतीत होता है निरुक्तकार यास्क\index{Yaska@Yāska} तथा जैमिनि समकालीन थे, क्यूंकि मीमांसासूत्र तथा निरुक्त में बहुत सी बातें समान दिखाई देती है, तथापि एक का दूसरें को परिचय था यह भास नहीं हाता है~। उदाहरणार्थ–निरुक्त\index{Nirukta@\textit{Nirukta}} \enginline{7.5} में मन्त्र तथा यागों में जिन देवताओं का उल्लेख किया है, उनके स्वरूप का विचार किया गया है~। उन्हीं बातों को मीमांसा सूत्र \enginline{9.1.6} से लेकर \enginline{9.1.10} तक विचार किया गया है~। 

\vskip 2pt

अन्य उदाहरण–जैमिनि सूत्र “भावार्थाः कर्मशब्दाः” (\enginline{2.1.1}) तथा यास्क\index{Yaska@Yāska} के “भावप्रधानम् आख्यातम्” (\enginline{1.1}) इन दोनों सूत्रों के विचार में अत्यन्त समानता दिखाई देती है~। जैमिनि कहते है ‘सर्वेषां भावोऽर्थो इति चेत् येषामुत्पत्तौ स्वे प्रयोगे रूपोपलब्धिः तानि नामानि, येयां तूत्पत्तावर्थे स्वे प्रयोगी न विद्यते तान्याख्यातानि~। (\enginline{2.1.2.} से \enginline{4})

यास्काचार्य कहते है ‘तद्यत्रोभे भावप्रधाने भवतः पूर्वापरीभूतं भावमाख्यातेनाचष्टे व्रजति पचतीत्युुपक्रमप्रभृत्यपवर्गपर्यन्तं मूर्त्तं सत्वभूतं सत्वनामिर्व्रर्ज्या पक्तिरिति~। (\enginline{1.1}) एवञ्च पूर्वपक्षी याज्ञिक ने वेद को अर्थहीन बताया है, इसके समाधानार्थ जैमिनि तथा यास्क ने समर्थ रूप से खण्डन किया है, जैमिनि ने नव सूत्रों में (\enginline{1.2.31} से \enginline{39}) तक पूर्वपक्ष का खण्डन किया है, उनमें से पाँच सूत्र यास्काचार्य के समान है~। तथा मीमांसा के उन्ही नव सूत्रों में से और दो सूत्र “अविद्यमानत्वात्” तथा “अचेतनेऽर्थ बन्धनात्” इन दो सूत्रों में यास्क के ‘अनुपपन्नार्था भवन्ति' इस सूत्र में देखा जा सकता है~।

यास्क तथा जैमिनि\index{Jaimini} के सिद्धान्त समरूप है, यदि जैमिनि यास्क\index{Yaska@Yāska} के परवर्ती होते तो निश्चित रूप से उनके सिद्धान्तों का अध्ययन किया होता, तथा उनका उल्लेख मीमांसासूत्रों में दिखाई पड़ता, परन्तु न यास्क का उल्लेख करते है न यास्क जैमिनि का~। इससे यह ज्ञात होता है कि दोनो प्रायः समकालीन थे~। सभी विद्वानों ने एकमत से स्वीकारा है कि यास्क\index{Yaska@Yāska} बुद्ध\index{Buddha, the} के पूर्ववर्ती थे लगभग \enginline{500} ईसा पूर्व अतः जैमिनि\index{Jaimini} भी यास्क के समकालीन होने से बुद्ध के पूर्ववर्ती सिद्ध होते है~। निष्कर्ष यह है पॉलॉक के आक्षेपो का कोई आधार नहीं है~।


\section*{उपसंहार}

\vskip 8pt

\begin{verse}
वेदा वा एते, अनन्ता वै वेदाः~।\\कृत्स्न एव हि वेदोऽयं परमेश्वर गोचरः~।\\भूतं भवत् भविष्यच्च सर्वं वेदात् प्रसिध्यति~।
\end{verse}

\vskip 6pt

मीमांसाशास्त्र में यागादिरूप वेदार्थ का प्रतिपादन किया गया है अतः वेदवाक्यों के अवगमनार्थ इस शास्त्र का अद्वितीय योगदान है~। शेल्डन पॉलॉक अपने कई लेखनों में इस विचारधारा को सदोष दर्शाते है समाज के कई विकृतियों का कारण यह मीमांसाशास्त्र है ऐसा भी वो कहते है~।

अतः अपने इस लेखन में हमने पॉलॉक के आक्षेपो का समाधान करने का प्रयास किया है~। सर्वप्रथम हमनें धर्माऽधर्म\index{dharma@\textit{dharma}} के विवेक में एकमात्र वेद को प्रमाण बताया है तदनन्तर वेद की अपौरुषेयता को युक्ति तथा तर्क के माध्यम से प्रतिपादित किया है~। परन्तु शेल्डन पॉलॉक यह कहते है हि अपौरुषेयता औत्पत्तिक इत्यादि वास्तविक में वेदों का स्वधर्म नहीं है, बौद्धों के आक्षेपो के प्राप्त होने पर मीमांसको के द्वारा सिद्ध किया तर्क है~। 

अतः अपने इस पेपर में हमनें सम्यक्तया इस बात को समाधान करने का यत्न किया है~। अतः आशा करता हुं कि मेरा यह पेपर पॉलॉक के प्रश्नों का समाधान करने के लिए एक साधन है~।


\bgroup

\selecteng

\section*{Bibliography}

\retainauthsanskrit

\begin{thebibliography}{99}
\bibitem{chap1–key1} Arnason, Johann P., Eisenstadt, S.N., Wittrock, Björn. (2005). \textit{Axial Civilizations and World History}. Brill. Leiden.

 \bibitem{chap1–key2} Acarya, Vedavrata (ed). (1962). \textit{Mahābhāṣya} of Patañjali with \textit{Pradīpa} of Kaiyaṭa and Uddyota of Nāgeśa. Gurukul Jhajjar. Rotak.

 \bibitem{chap1–key3} Devasthali, G.V. (1939). ‘On the probable date of Jaimini and his sūtras’. \textit{Annals of the Bhandarkar Oriental Research Institute}. Vol. 21, No. 1/2 (1939––40). pp~63––72.

 \bibitem{chap1–key4} Jha, Bakshi, Jha, Mukund (ed). (1989). \textit{Nirukta of Yāska with Vivṛti}. Chowkambha Sanskrit Pratishthan. Delhi.

 \bibitem{chap1–key5} \textit{Mahābhāṣya of Patañjali}. See Vedavrata Acarya.

 \bibitem{chap1–key6} \textit{Mīmāṁsāsūtras of Jaimini}. See Meheshchandra.

 \bibitem{chap1–key7} Meheshchandra, Nyayaratna (ed). (1873). \textit{Mīmāṁsāsūtras of Jaimini with Śabarabhāṣya}. Asiatic Society of Bengal. Calcutta.

 \bibitem{chap1–key8} \textit{Nirukta} of Yāska. See Bakshi and Mukund Jha.

 \bibitem{chap1–key9} Pollock, Sheldon. (1989).‘Mimamsa and the Problem of History in Traditional India. \textit{Journal of the American oriental society}. Vol. 109, No.~4 (Oct. –– Dec., 1989). pp 603––610.

 \bibitem{chap1–key10} Pollock, Sheldon. (2006). \textit{The Language of the Gods in the World of Men}. University of California Press. California.

 \end{thebibliography}

\egroup

\egroup

