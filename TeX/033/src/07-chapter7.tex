
\chapter{Saṁskṛti in Context}\label{chapter7}

\Authorline{Charu Uppal\supskpt{*}}

\begin{flushright}
\textit{(ucharu@gmail.com)}
\end{flushright}


\section*{Abstract}

The \textit{Rāmāyaṇa}\index{Ramayana@\textit{Rāmāyaṇa}} is the older of the two Indian epics that has inspired essays, commentaries, movies, TV series, stage performances, music and poetry for centuries. Characters of the \textit{Rāmāyaṇa} have lived on for centuries in the hearts of the Indian masses, and come alive even today when generation after generation children are named after its protagonists, or when the \textit{Rāmāyaṇa pāṭha} (recitation of the entire epic) is observed in homes as a purifying ceremony, at occasions such as weddings, birth of child, and before moving into a new house. In a country that makes over a thousand films every year, enactment of the \textit{Rāmāyaṇa} on makeshift stages (Rām-līlā\index{Ramlila@Rām-līlā}) still finds a sizable following every year, testifying to its place in the Indian psyche. It is in that context that this paper critiques Sheldon Pollock’s\index{Pollock, Sheldon} reading of the \textit{Rāmāyaṇa}. Using Rajiv Malhotra’s\index{Malhotra, Rajiv} work as the foundation, the paper elicits examples from both the texts and the lived experience in order to illustrate how Pollock misses reading the \textit{Rāmāyaṇa} in its own context, because he gazes at the epic from his pre-conceived bias that, though understands the meaning, does not resonate with terms such as \textit{śraddhā}\index{sraddha@\textit{śraddhā}} and \textit{itihāsa}\index{itihasa@\textit{Itihāsa}}. In addition, making use of Campbell’s\index{Campbell, J.} definition of myth, and its role in its respective society, the paper also highlights how Pollock’s\index{Pollock, Sheldon} understanding of the \textit{Rāmāyaṇa}\index{Ramayana@\textit{Rāmāyaṇa}} as a myth is constructed outside of Indian context. Furthermore, the paper is informed by the author’s association with a Rām-līlā committee that has run successfully for the last five decades by a community of volunteers, most of who hold white-collar jobs.

\begin{myquote}
It is true that no production of knowledge in the human sciences can ever ignore or disclaim its author’s involvement as a human being in his own circumstances, then it must be true that for a European or American studying the Orient there can be no disclaiming the main circumstances of this actuality: that he comes up against the Orient as a European or American first, and as an individual second. 

~\hfill Edward Said\index{Said, Edward} in \textit{Orientalism}
\end{myquote}

\begin{myquote}
Hinduism is not just a faith. It is the union of reason and intuition that cannot be defined but is only to be experienced. Evil and error are not ultimate. There is no Hell, for that means there is a place where God is not, and there are sins, which exceed his love. 

~\hfill Dr. Radhakrishnan\index{Radhakrishnan, S.}
\end{myquote}

\begin{myquote}
Celebrating civilization perfection is nothing more than a blind abdication of self-criticism. 

~\hfill (Pollock, as cited in Gould\index{Gould, R.}, 2008:533)
\end{myquote}


\section*{Introduction}

To this day we call them Ram Uncle and Ravan Uncle. We usually do not remember their names because Ram and Ravan were the roles they played on the local stage, year after year. And, we saw them mainly between July and October of every year, when they rehearsed, slipped into their respective roles and finally embodied those roles. From July of every year, after the first \textit{pūjā}, called \textit{Gana Bandhan}\supskpt{1}, most if not all of the actors and those associated with the Rām-līlā\supskpt{2} would follow a vegetarian diet, refrain from alcohol, sleep on the floor and keep a low key in their social activities, until the day of the Dusshera when Ravan would be killed on stage, and after which the actors could resume their regular life. During the ten-day performance they became their roles, in the way they walk, the way they speak and the way they live. So much so, that on the last day of the performance, when Ram Uncle returned home, I was told, that his parents would do an \textit{ārti}\supskpt{3}, like the one offered to a deity in a temple, for he had embodied the part/qualities of Lord Ram for the last six months.

To an outsider, who neither understands nor feels what a Hindu would naturally feel when Lord Ram’s name is spoken, worshipping one's own son is nothing but the continuation of the drama, and at worst, a superstitious activity. To the uninitiated into living a truth, the ritual of \textit{ārti} would seem pointless and without the sacred. Simply put, very few who have not lived or grown up with the concept of \textit{bhakti}\index{bhakti@\textit{bhakti}} can understand how a parent turns into a devotee, and the son into a God. But as Campbell\index{Campbell, J.}, who emphasizes the place of heart over rational faculties when a myth is lived, explains, ‘a ritual is an enactment of a myth, - insofar, the myth is a revelation of dimensions of your own spiritual potential, you are activating those dimensions in yourself and experiencing them" (Maher\index{Maher, J. M.} and Briggs\index{Briggs, D.} 1990:35)\supskpt{4}. In that context, performing an \textit{ārti} for their son, Ram Uncle’s parents were living the core of the Hindu philosophy, that the divinity is defined by its ‘\textit{bhāva}’ (attitude and quality), and neither is restricted to any one form nor are other-worldly. And the performance of Rāmlīlā\index{Ramlila@Rām-līlā} for centuries is a yearly ritual that allows the audiences to participate in an enactment of itihāsa\index{itihasa@Itihāsa} as it informs their daily lives.

This paper critically evaluates Professor Sheldon Pollock’s\index{Pollock, Sheldon} take on the \textit{Rāmāyaṇa}\index{Ramayana@\textit{Rāmāyaṇa}}, and its impact on Indian cultural and political life, with special focus on gaps in Pollock’s methodology and sampling, which, due to their inconsistencies, would lead us to erroneous conclusions.


\section*{Why Pollock?}

Presently, Professor Sheldon Pollock is one of the most revered professors of Indology, who has pointed at the \textit{Rāmāyaṇa} among other texts, as a text that is guilty of using language, story and characterization as tools to demonize the non-Hindus, particularly Muslims, and has resulted in much violence against minorities, especially Muslims, again Pollock’s study of the \textit{Rāmāyaṇa} is essentially a study to link its plot, language and texts to power structures and put a political rather than a sacred lens on the epic. Pollock considers the \textit{Rāmāyaṇa} a political tool to label all non-Hindus, especially the Muslims as the ‘other’, as one which has led to violence against the Muslim community (Pollock 1993). He attributes power and hegemonic\index{hegemonic discourse} structures to both the plot and characters of the great epic\supskpt{5}.

Scholars have contended that (for this very reason), Pollock’s\index{Pollock, Sheldon} work will likely “play a dominant role in shaping the wider public image of pre-modern Indian, especially Sanskrit, language, and culture along with the forms of polity related to them, for years if not decades to come.” (McCrea\index{McCrea, L.} 2013:117; Gonzalez-Reimann\index{Gonzalez-Reimann, L.} 2006:204\supskpt{6}). Some even suggest that Pollock’s scholarship on India is “scholarship about the world” (Gould\index{Gould, R.} 2008:557) and that it “can and should transform contemporary understandings for the relationship between culture and power, the status of literature, and the state, ethnicity, and polity throughout history” (Gould 2008:534).

It would be unfortunate if a thesis of such significance as Pollock’s, were founded on a methodology that use lenses and ideology that are far from appropriate for the subject used deconstruction. Especially since Pollock also, through his thesis, seems to have established that modernity or ‘newness’ (Gould 2008) can emerge by desacralizing the Indian texts (\textit{śāstra}-s\index{sastra@\textit{śāstra}}) without which they are incapable of being interpreted in a novel manner.

In this paper, I attempt at illustrating how this aspect of ‘desacralizing’ is a major flaw in Pollock’s methodology, as the ground reality demonstrates. In desacralizing the texts, Pollock steps outside of the texts and views them as ‘he wishes to see them’ ignoring how they are regarded by those who have lived with the heritage for centuries, and ignores several crucial aspects which can challenge his ideas, in order to establish his pre-conceived theory; rather than allow the theory to emerge as a result of the scholar both regarding the text as those who live them; and then corroborating the evidence with the ground reality in India. The following sections first explain the difference between concept of myth and \textit{mithyā}\index{itihasa@\textit{Itihāsa}}, and \textit{itihāsa} and history, establish the \textit{Rāmāyaṇa}’s\index{Ramayana@\textit{Rāmāyaṇa}} role in Indian cultural life, and discuss how Pollock’s\index{Pollock, Sheldon} method of desacralizing is flawed because it fails to view the gestalt of \textit{itihāsa}, and how \textit{itihāsa} is lived through yearly performance of the \textit{Rāmāyaṇa}.


\section*{Myth, Mythos, \textit{Mithyā} \& \textit{Itihāsa}}

While myth is often implied to mean a lie, a fiction, or something untrue, as its derived meaning from the Greek word \textit{mythos}, scholars have contested that meaning in the context of the Sanskrit word ‘\textit{mithyā}’ which implies a reality in between truth (history) and untruth (myth), and points towards a reality beyond our worldly understanding. Joseph Campbell\index{Campbell, J.}, the world-renowned mythologist considers myths as clues, which ‘direct us towards the experiencing the spiritual potentialities of the human life’ (Moyers\index{Moyers} 1990:5) for myth is a metaphor that is indicative of spiritual powers that lie within us (Maher\index{Maher, J. M.} \& Briggs\index{Briggs, D.} 1990).

Myths are narratives with multiple meanings that hold sacred value for the respective cultures and are carried out through their rituals. Therefore, these narratives are considered to be true from within the respective faith systems, and when regarded in context, lend themselves to expressing respective systems of thought and values. Although it is important to recognize that myths are usually regarded metaphorically and not literally, so myths can be both rooted in history and be fictitious (Carpentier’s\index{Carpentier, C.} Lectures on the Website, accessed May 15, 2016) e.g. Sun worship is not about worshiping the heavenly body as much as an acknowledgement of its life-giving quality to the entire planet. Which, it must be emphasized, has not changed since time immemorial. Therefore an ancient ritual of Sun worship is also an indication of ancient humans’ knowledge, however subconscious, of the influence of the Sun on our planet. Similarly Rāma and Rāvaṇa\index{Ravana@Rāvaṇa} are qualities that bring us close to or distance us from the divine.

\textit{Myth}, in the West, is used as the diametric opposite of history. But Rajiv Malhotra\index{Malhotra, Rajiv} emphasizes, in his path-breaking book, \textit{Being Different} (Malhotra 2013), that myth ‘uses fiction (story\supskpt{7}) to convey truth’ (Kindle Edition, Location 1138), and can be enacted out via a ritual (Myers 1990). In addition, to contrast it with the frozen idea of history, as in the West, the Indian word for history ‘\textit{itihāsa}’\index{itihasa@\textit{Itihāsa}}, sometimes translated as myth by those studying Indian texts, comprises both history and myth (Malhotra 2013). But the schism in the way the West sees others and itself is exemplified in \textit{Being Different}, where Malhotra shares a story about a Journalism professor’s struggle to include myths of Western civilizations in a class on ‘World Mythology’.

\begin{myquote}
“Western scholars unable to deal with the multiple renditions of \textit{itihāsa}, tend to categorize it all as myth, and myth alone. Their own myths are recounted as history. Indian spiritual texts are subject to interpretive methods, which are entirely different from those used to study the tales of Jewish and Christian religions. For example, the West is studied using sociological methods and tools, whereas so-called primitive societies through anthropology and folklore; European and American social units are always described as communities, never tribes.”

~\hfill Malhotra\index{Malhotra, Rajiv} (2013: L1139)
\end{myquote}

Since the West’s own myths are taught as history, and the West does not have a category similar to \textit{itihāsa}\index{itihasa@\textit{Itihāsa }}, Malhotra (2013) argues that Indologists are prone to misinterpretations when they use a Western lens to study \textit{itihāsa}, which is more concerned with truth rather than history, and can be told through multiple perspectives - hence it is that we have myriads of \textit{Rāmāyaṇa}-s.\index{Ramayana@\textit{Rāmāyaṇa}}

\begin{myquote}
“Parables abound in dharmic scriptures, too, but these inspire by the lessons they teach and not by claims of being the exact records of historical events. Hindus participating in rituals in temples do, for the most part, follow a received and codified tradition, and a minority might believe in the narratives they celebrate as literally having happened. Most Hindus tend to view the historical events in a fluid manner.”

~\hfill Malhotra (2013: L1110)
\end{myquote}

Malhotra continues to highlight that history, in the context of \textit{itihāsa}\index{itihasa@\textit{Itihāsa}}, is for the Hindus an instructional (and not a constitutive, as Pollock\index{Pollock, Sheldon} posits (1984:508)) tale which can be superseded by embodying the truth that the parable instructs to teach. For, a \textit{dharma}\index{dharma@\textit{dharma}} practitioner who studies \textit{itihāsa} -

\begin{myquote}
“..explicitly aspires to bring about a change within, emphasizing the virtues illustrated in the narratives and not the historical facts. Lord Rāma and Lord Krishna are embodiments of \textit{bhavas} (attitudes) and their historical significance is superseded by the values they convey.”

~\hfill Malhotra (2013: L1114)
\end{myquote}

While Pollock acknowledges the power of myth, he interprets the way he chooses to (Pollock 1984:508)\supskpt{8}. In that context, Pollock starts on a wrong note, when he limits his understanding of Rāma and Rāvaṇa\index{Ravana@Rāvaṇa} merely as good and evil, divine and demonic. In fact, he attributes harmful intentions in upholding Rāma as the model King, an ideal man, when he suggests that the \textit{Rāmāyaṇa}’s\index{Ramayana@\textit{Rāmāyaṇa}} sole purpose has been in creating and demonizing the ‘other.’ (Pollock 1984; Pollock 1993). Furthermore, Pollock demonstrates his inability to grasp the concept of \textit{itihāsa}\index{itihasa@\textit{Itihāsa}} as he uses the word Euhemerization\index{Euhemerization}\supskpt{9}(ascribing historical basis to mythology) for the divinity of Rāma (Pollock\index{Pollock, Sheldon} 1984: 506). The \textit{Rāmāyaṇa}\index{Ramayana@\textit{Rāmāyaṇa}}, for Hindu society, is a metaphor, although Pollock treats it like a mystery, which he has attempted to uncover through his writings. If there is a mystery to the \textit{Rāmāyaṇa}, as a practitioner, the author can state, that it is this, that taking the example of Rāvaṇa\index{Ravana@Rāvaṇa}, we can understand that most venerated of scholars can be proven wrong about the use of knowledge, and yet in academia today, scholars hold mind over heart and sometimes ethics, in the pursuit of understanding a system.


\section*{The \textit{Rāmāyaṇa}: Popularity and Presence in the Indian Psyche}

Considered as the \textit{ādi-kāvya} (the First Poem) the \textit{Rāmāyaṇa} is one of the oldest and longest epics in world literature (Embree 1972). Pollock\index{Pollock, Sheldon} is correct in estimating the impact of the \textit{Rāmāyaṇa} and its characters on the Hindu mind and culture. The \textit{Rāmāyaṇa}, written as \textit{itihāsa}\index{itihasa@\textit{Itihāsa}}, is not merely a story, but an allegory that instructs and guides, through the use of philosophy and ethics, in matters of the ‘ideal’ — what constitutes an ideal king, ideal husband, ideal father, ideal wife (Embree\index{Embree, A. T.} 1972). Children are named after its characters, numerous television series and movies have been created depicting the \textit{Rāmāyaṇa}, every year thousands of Rāmlīlās\index{Ramlila@Rām-līlā} are performed around India and many other countries, and the recitation of the entire \textit{Rāmāyaṇa} is used also as a purification ceremony for various occasions from pre-birth to post funeral rites. Interestingly, it is the \textit{Rāmāyaṇa} that is given as a gift to newly-weds as a reminder of the life of an ideal couple, as opposed to the \textit{Mahābhārata}\index{Mahabharata@\textit{Mahābhārata}}, which is not held in the same regard\supskpt{10}. It is after all, the greatest and most ideal love story of Gods who incarnated as an ideal couple, ideal human beings to act as models — to play out the \textit{Līlā}\supskpt{11}.)

Pollock is also correct, although incomplete, in saying that Rāvaṇa\index{Ravana@Rāvaṇa} is not viewed as the most positive character to emulate. However, as will be explained later that this does not equate to demonization of a character addressed as \textit{Daśānan}, the one with (knowledge of) ten heads. Neither is there any link, as will be illustrated through this paper, between widespread impact of and love for the \textit{Rāmāyaṇa}\index{Ramayana@\textit{Rāmāyaṇa}} and the violence against the non-Hindus.


\section*{Our Rāmlīlā\index{Ramlila@Rām-līlā}}

Rāmlīlā\index{Ramlila@Rām-līlā}, the divine play, through which Rāma’s\index{Rama@Rāma} life is told is enacted out as a ten day ritual throughout India culminating in a big celebration on Vijayadaśamī or The Victorious Tenth (lunar day), the day effigies of Rāvaṇa\index{Ravana@Rāvaṇa} are consigned to flames. While this act might seem like, to someone like Pollock\index{Pollock, Sheldon}, another indication of the demonization of Rāvaṇa, the act of burning of the effigies, like a ritual, allows the audience to contemplate on the battle of good vs. evil that too often rages in our own hearts. It would surprise Pollock to know then that the people vie for the left over bamboos from the effigies, for it is considered sacred to keep the ‘bones’ of Rāvaṇa, the wise man, in the house.

Rituals such as these, once known, can question all of theorization that Pollock has created to impose his view of looking, rather than understanding the complex place of good and evil, right and wrong in Hinduism. While Hindus revere Lord Rāma for being the \textit{puruṣottama}, the ideal man, their love is only increased for the butter-stealing Krishna. While the lived reality of Hinduism is too complex to be explained and understood by those who do not practice it, for it combines both the sacred and the secular, what is surprising is Pollock’s absolute contempt, as evident in the following comment, for the tradition, that attributes divinity to all beings:

\begin{myquote}
“Much of the argument against the divinity of Rāma, furthermore, is based on a sense of the “divine" that conceals an embarrassingly narrow and un-selfreflective ethnocentricity, and on the use of an inapplicable set of critical canons.”

~\hfill (Pollock 1984:508)
\end{myquote}

Throughout this paper there will be references made to one such Rāmlīlā that has been staged in Delhi for nearly seven decades. Rajendra Nagar Arts Club, formally known as Shri Rāmlīlā Committee, had humble beginnings in 1949 when a group of people started a local club that decided to stage a Rāmlīlā every year. All the actors were from the neighborhood, and the nominal entrance fee that was charged funded the project. Over the years, the club has become a hallmark of the locality, and the Rāmlīlā still draws large crowds. Throughout the years, though professional actors were hired to play some roles, the performances have been sustained by children and grandchildren of the initiates of the project. Most of the actors hold a day job, but are dedicated to the continuation of Rāmlīlā\index{Ramlila@Rām-līlā}, which since 1980 has dropped the entrance fee. Some years, attempts were made to broadcast it live through the local channel for the benefit of the elderly and those who could not attend in person.

An interesting aspect of Our Rāmlīlā\index{Ramlila@Rām-līlā} was that during its ten-day performance, usually one or more social plays completely unrelated to the \textit{Rāmāyaṇa}\index{Ramayana@\textit{Rāmāyaṇa}} were included to focus on issues such as promoting rural development and questioning dowry practices. Over the years, due to reduction in the number of holidays given to schoolchildren, social plays have been dropped. This inclusion of completely secular plays in a very sacred performance of the \textit{Rāmāyaṇa} is an indication of self-reflexive ability of Hinduism and an indication that secular and sacred mingle in India on a regular basis.

The script used by the club is an amalgam of several \textit{Rāmāyaṇa}-s and is especially known for the lessons that Rāvaṇa gives to Lakṣmaṇa as he lies dying. In an email communication, the author was apprised that the script has not remained static and has been improvised, modified several times since 1949, testifying to the freedom allowed in adapting the epic, as is the case with Indian storytelling, no matter how many times they are told:

\begin{myquote}
“The precise story of Rāma\index{Rama@Rāma} can never be produced, and each attempt involves a combination of reproduction (by supplier), re-narration (often interactive), and re-presentation (by audience). Thus itihasa\index{itihasa@\textit{Itihāsa}} changes, evolves and adapts to circumstances as per the prevailing consensus.”

~\hfill Malhotra\index{Malhotra, Rajiv} (2013: L1167)
\end{myquote}

In addition, to give actors a break, Sahitya Akademi actors are invited one day, to play a one-Act \textit{Rāmāyaṇa}. Rajendar Nagar Arts club has become a classic example of \textit{dharma}-s\index{dharma@\textit{dharma}} and \textit{dharma}-related activities being intertwined with social life. True to its \textit{dhārmic} nature of ‘upholding’ a community, Rāmlīla’s stage is not just about one story but for the ten days it becomes a collective stage to display local talent. In between scene-changes, children and adults alike are allowed to exhibit their talent in poetry recitation, skits, stand-up comedy, singing and /or even dance to a Bollywood song. Wonder what Pollock\index{Pollock, Sheldon} would say about the fact that the Muslim mega star Shahrukh Khan got his first audience at this stage during a scene change. Khan was also an enthusiastic ‘\textit{vānar}’ in Hanuman’s \textit{senā}.

Over the years, audiences of Our Rāmlīlā\index{Ramlila@Rām-līlā} witnessed that settings and performances have changed with changes in technology, but more importantly because of its ‘live’ nature, Our Rāmlīlā has remained unpredictable. The \textit{Rāmāyaṇa}\index{Ramayana@\textit{Rāmāyaṇa}} as Rāmlīlā is the best example of the fluidity of \textit{itihāsa}\index{itihasa@\textit{Itihāsa}}. Some scenes are exaggerated and get laughs; especially the one where the Hanumān on the stage in the process of desecrating the Ashok Vāṭikā hurls bananas at the audience, making it an interactive play. Or when Kumbhakarṇa is being woken up from his deep sleep slumber, and some \textit{sainik}-s fall into the audience. Every year has a different flavor.


\section*{Pollock’s\index{Pollock, Sheldon} Out-of-Context Methodology}

As explained above, the paper illustrates through the examples how Pollock’s analysis of the \textit{Rāmāyaṇa} is not situated in the context of Hindu society, emotions, feelings and understanding of the epic and therefore must be evaluated.

\textit{Desacralizing-Decontextualization:} Pollock’s strategy of forwarding this theory of ‘aestheticisation of power’ is to first desacralize the texts that he studies (Malhotra\index{Malhotra, Rajiv} 2016a:L3462). In desacralizing the texts, Pollock commits academic blasphemy by divorcing the object of his study from its context as he tries to understand its impact. It is akin to the Californian version of understanding \textit{karman}\index{karman@\textit{karman}} without a belief in reincarnation. Any qualitative research, especially one that deals with the understanding of systems, cultures and texts of cultures that one does not practice, or live with, has to be grounded in a research methodology that is appropriate for the topic and attempt to study its object being as close to its lived reality.

Appropriate application of research methods is crucial to a thorough consideration of the topic of study. The concept of \textit{Rāma-rājya}\index{rama-rajya@\textit{Rāma-rājya}}, inspired by the \textit{Rāmāyaṇa} is a prime example of a call to rulers and officers to sacrifice their personal motives for the larger good—by following the \textit{dharma}\index{dharma@\textit{dharma}} of a king, who must be just. It is this aspect of the \textit{Rāmāyaṇa} that guides, instructs through allegories for the \textit{dharma} practioner who reads it with ‘\textit{śraddhā}’\index{sraddha@\textit{śraddhā}} and ‘\textit{bhakti}’\index{bhakti@\textit{bhakti}} that Pollock\index{Pollock, Sheldon} is missing in his lens.

Desacralization therefore, amounts to de-contexutalization of studying the \textit{Rāmāyaṇa}\index{Ramayana@\textit{Rāmāyaṇa}}, which then, according to Pollock, is merely literature – freed from the “clutches of sacredness”. Pollock\index{Pollock, Sheldon} uses Vico’s\index{Vico, Giambattista} theories as the reasoning for removing the sacredness from \textit{śāstra}-s (Malhotra\index{Malhotra, Rajiv} 2016a:L1366 \& L1371) and therefore \textit{itihāsa}\index{itihasa@\textit{Itihāsa}}, which uses storytelling to convey the teachings of \textit{śāstra}-s\index{sastra@\textit{śāstra}}. Not surprisingly, Vico’s ideas are very ethno(Euro)-centric and can be hardly considered universal. Vico places the philosophical, post-rational thinking, characterized by European (and by analogy, the West) societies, over what he considers poetic thinking, found in primitive societies (Malhotra 2016a: L1550).

\begin{myquote}
“There are two key points in his lens, which are inspired by Vico: 1. The principle of treating the secular as separated from the transcendent, and the view that ancient texts and thinkers were pre-rational, mythically oriented, and emotional – and they lacked the rationality to develop and apply this principle to look at history clearly, a history driven by purely material acts.” 

~\hfill (Malhotra 2016a:L1656)
\end{myquote}

However, Christians\index{Christians, C.} and Carey\index{Carey, J.} (1989) argue against some of the ways of viewing the subject matter as suggested by Vico\index{Vico, Giambattista} and those influenced by him (p. 355), and call for a more comprehensive approach. They question the ‘natural science model of the social sciences’, (p. 355) and argue against the idea that social sciences, like natural sciences are said to “…develop laws that hold irrespective of time and place, to explain phenomenon through causal and functional models, to describe relationships among phenomena in essentially statistical and probabilistic terms” (p. 354). Instead, placing a higher value on symbols and context, Christians and Carey provide four criteria that make for a valid and a thorough consideration of the topic of study: naturalistic observation, contextualization, maximized comparisons, and sensitized concepts (Christians and Carey 1989).

Naturalistic observation would require a researcher to understand the symbols and their context as the Hindus do:

\begin{myquote}
“Symbolic activity is recognized as central to our personal and social experiences; symbols and symbolic patterns the irreducible socio-cultural data of qualitative research, getting an insider’s view takes it for granted that to understand someone’s thought one needs to think with the same symbols. The social scientist must study the human spirit as expressed through symbolic imagery. ‘The Chicago School’ taught us that social feelings (attitudes and sentiments) lifestyle are most fully expressed in actual situations, and must be recovered unobtrusively through participant observation, from personal documents, and by open-ended interviewing. To get inside the realms of lived experience, the natural processes of communication are especially valuable (such as correspondence, eyewitnesses accounts, songs, jokers, folklore, memoranda, diaries, ceremonies, citizen group reports, sermons), and methods must be avoided that disrupt the social process and thereby skew our vision.” 

~\hfill (Christians\index{Christians, C.} and Carey\index{Carey, J.} 1989:361)
\end{myquote}

While Pollock\index{Pollock, Sheldon}, an esteemed and thorough scholar can be lauded for his vast scholarship, he fails to grasp the ideals that Hindus are asked to learn from the epic, essentially because he ignores how the text is being lived in daily lives whether in India or even Muslim majority countries like Indonesia. And once the concept of \textit{pāramārthika}\index{paramarthika sat@\textit{pāramārthika sat}} (the transcendent/sacred) is done away with, one cannot fathom the power of the epic in evoking qualities of justice, love, duty, and self-sacrifice among its followers.

Neither does Pollock pass the test of the second criteria, viz. contextualization, which we consider here:

\begin{myquote}
“- a significant tool that allows a researcher to understand how his/her subject feels about and understands a phenomenon. As culture and related symbols are complex and usually have multilayered meanings, it is important to understand a text, an event, or a ritual, in a larger multi-dimensional context.
\end{myquote}

\begin{myquote}
Contextualization is a vital dimension of interpretive studies. While extraordinarily complex, the guideline calls our attention to immediate, wider cultural and historical context if we are to interpret human interpretation accurately.” 

~\hfill (Christians and Carey 1989:361)
\end{myquote}

While Pollock implies that widespread use of the \textit{Rāmāyaṇa} can be owed to its use as a political tool, his analysis does not consider the idea that, in the Indian context, the rise and decline in worship of various \textit{avatāra}-s, according to the \textit{yuga}-s has been common. When he suggests that characters of the \textit{Rāmāyaṇa}\index{Ramayana@\textit{Rāmāyaṇa}} behave as puppets as though without any will, he is implying that the followers of epic might be driven in the same robot-like manner to demonize non-Hindus and those of lower castes. But he conveniently ignores characters such as Khevaṭ and Śabarī who are not only from the lower strata, but who Rāma\index{Rama@Rāma} expresses gratitude to, for helping him in his journey. In addition, Pollock overlooks factors such as Rāvaṇa\index{Ravana@Rāvaṇa} himself being a learned Brahmin, whose father was the venerated Sage Viśravas\supskpt{12}. Why does the \textit{Rāmāyaṇa} create the other of a brahmin? Especially when Pollock\index{Pollock, Sheldon} implies that it was with the collaboration of the brahmins that the kings/rulers demonized the outsiders. Also overlooked is the fact that the author Vālmīki\index{Valmiki@Vālmīki} himself is supposed to be not of a higher caste, and Rāvaṇa\index{Ravana@Rāvaṇa} a revered, learned one of a higher caste and a king, the very people Pollock\index{Pollock, Sheldon} suggests oppressed the ‘lower castes’?

If Pollock\index{Pollock, Sheldon} were interested in considering the context, he would have visited Our Rāmlīlā (or any Rāmlīlā\index{Ramlila@Rām-līlā} for that matter), and interviewed (as this author has done) many of its actors. The one who plays Vibhīṣaṇa proudly stated that his favorite character and \textit{guru} is Rāvaṇa, who he considers ‘wise, intelligent, a researcher, resourceful enough to make accessible the means to his \textit{mokṣa}\index{moksa@\textit{mokṣa}}, someone with foresight and power’\supskpt{13} (Personal interview with the author, 2009). To corroborate this interview, the actor was contacted again in August of 2016 via email. The response he gave not only confirmed his earlier view, but he also stated that he has not only read the \textit{Rāmāyaṇa}\index{Ramayana@\textit{Rāmāyaṇa}} in its entirety, but also texts purportedly authored by Rāvaṇa such as \textit{Uḍḍīśa Tantra.}

It is important to mention here that on grounds that the text is sacred, it is not above criticism. Several essays, jokes and stand-up comedians have questioned, critiqued and laughed at the characters of \textit{Rāmāyaṇa} without any restriction or condemnation thereby questioning the notion that the text has not been allowed to be critiqued. But instead, Pollock brings his own theories, provides slim, random and unsystematic evidence and weak links that do not consider the context of \textit{śraddhā}\index{sraddha@\textit{śraddhā}} which is how a Hindu approaches the \textit{Rāmāyaṇa}, to support his ideas.

In context however, Pollock may arrive at what Hindus feel and have been taught through the \textit{Rāmāyaṇa}, if he uses the sacred lens. But since he insists on using the lens that has been provided by an ideology that proclaims ‘religion is the opiate of the masses’, he is incapable of arriving clearly at the impact that the \textit{Rāmāyaṇa} has had, and his claims seem meaningless. The \textit{Rāmāyaṇa}, as discussed above is \textit{itihāsa}\index{itihasa@\textit{Itihāsa}}, both myth and history written in a form so that it can be used by people in their daily lives (\textit{vyāvahārika}\index{vyavaharika sat@\textit{vyāvahārika sat}} level) to connect with the transcendent (\textit{pāramārthika}\index{paramarthika sat@\textit{pāramārthika sat}} level).

Another dimension of the interpretive process, as suggested by Glaser and Strauss (as cited in Christians\index{Christians, C.} and Carey\index{Carey, J.} 1989) is to maximize the comparisons (Glaser and Strauss 1989:366). If Pollock\index{Pollock, Sheldon} wanted to strengthen his argument of demonization of the ‘non-Muslims’ he could have compared other countries such as Thailand and Indonesia where the \textit{Rāmāyaṇa}\index{Ramayana@\textit{Rāmāyaṇa}} still forms an integral part of cultural experience, there are the temples dedicated to Rāma\index{Rama@Rāma} and Rāmlīlā\index{Ramlila@Rām-līlā} is performed regularly. If the text had such universal values, it would be interesting to see Pollock’s\index{Pollock, Sheldon} ideas on how the epic is received in other Asian countries.

And finally, Pollock fails on the last criteria, viz. ‘sensitized concepts.’ That is:

\begin{myquote}
“Formulating categories that are meaningful to the people themselves, yet sufficiently powerful to explain large domains of social experience. Interpretive research seeks to capture original meanings validly, yet explicate them on a level that gives the results maximum impact.” 

~\hfill (Christians\index{Christians, C.} and Carey\index{Carey, J.} 1989:370)
\end{myquote}

Pollock problematizes the divinity of Rāma, a concept taken for granted by the Hindus, by stating that unlike the way it is revealed, Vālmīki\index{Valmiki@Vālmīki} intended Rāma’s divinity from the very outset\supskpt{14}, and this is obvious by the fact that no one questions Rāma’s divinity.\supskpt{15} However, Gonzalez-\index{Gonzalez-Reimann, L.}Reimann(2006:207)\supskpt{16}, argues that although Vaiṣṇava commentators, due to their love for the God King, cannot be expected to question the divinity of Rāma\index{Rama@Rāma}, yet unlike what Pollock thinks, this question has been asked in several instances, including the \textit{Adhyātma Rāmāyaṇa}\index{Adhyatma Ramayana@\textit{Adhyātma Rāmāyaṇa}}. By questioning whether or not Rāma is aware of his being an \textit{avatāra} of Viṣṇu, Pollock questions the very basis for which the \textit{Rāmāyaṇa} is popular among the masses. It is obvious that he does not understand the term, ‘\textit{Līlā}’ (divine play), that one word that explains so much for the Hindus. For the Hindus, Rāma is considered a ‘\textit{tāraṇ-hār}’ the one who delivers \textit{ātman} across the ocean of \textit{saṁsāra}, however, whether or not he is aware of his divinity is not an important feature of the \textit{Rāmāyaṇa}. As stated elsewhere in the paper, the epic is a metaphor, not a mystery\supskpt{17}.

Furthermore, in context if Pollock had analyzed the performance of the \textit{Rāmāyaṇa}, which is staged for the lay people, who then, according to him, get mobilized against non-Hindus, he would have found that unlike Aristotle’s\index{Aristotle} recommendations who is against portraying violence on stage, Bharata in \textit{Nāṭyaśāstra}\index{Natyasastra@\textit{Nāṭyaśāstra}} concerned more with style and beauty, allows/makes room for it, so long as it is presented aesthetically (Massey\index{Massey} 1992:62)\supskpt{18}.

Had Pollock\index{Pollock, Sheldon} visited ‘Our Rāmlīlā’\index{Ramlila@Rām-līlā}, he would have witnessed a very moving scene, not present in Vālmīki \textit{Rāmāyaṇa}\index{Ramayana@\textit{Rāmāyaṇa}} — Rāvaṇa’s\index{Ravana@Rāvaṇa} \textit{śikṣā} (teachings) to Lakṣmaṇa. Rāma’s equanimity and inclination towards utter justice and kindness is evident in how he deals with the dying King. During the act, when Rāma\index{Rama@Rāma} asks Lakṣmaṇa to seek life-advice from Rāvaṇa\index{Ravana@Rāvaṇa}, Rāma instructs him to stand at the feet of the teacher. That scene always draws the most emotional and profound silence in the audience.

Pollock has completely ignored considering the views of those who have read the \textit{Rāmāyaṇa}\index{Ramayana@\textit{Rāmāyaṇa}} and embodied its message. In the context of the Rāmlīlā\index{Ramlila@Rām-līlā} mentioned above, Rāvaṇa, a brahmin himself has been depicted as an intellectual, a learned man, well versed in various sciences of his time, but his fatal flaw is his ‘ego’, which is not redeemed by him being, either of a higher caste, or a king. This places Pollock outside of the context.


\section*{Insider/Outsider: Who is a Qualified Pupil?}

Malhotra\index{Malhotra, Rajiv} (2016a) talks extensively in his recent book \textit{The Battle for Sanskrit} about the concept of insiders and outsiders, a concept he adapts from anthropology. And Malhotra cautions, that an insider is not just a Hindu, but rather someone who understands, and is willing to see study Hinduism from the eyes of those who practice it. The categories of contextualization and sensitized concepts, which make for a thorough more honest, close to the real reading of a culture, would come naturally to an insider. Pollock fails this requirement.

In his methodology of desacralization and de-contextualization Pollock\index{Pollock, Sheldon} ignores both his own mentor and the qualities needed for being a student - as prescribed by a Vedāntic text called \textit{Vedāntasāra} (\textit{lit:} the essence of Vedānta) authored by Sadānanda. Previous scholars have referred to the need for a special mindset while approaching the \textit{śāstra}-s\index{sastra@\textit{śāstra}}, which should not be too enthusiastically academic or intellectual:

\begin{myquote}
“….One may read this translated text precisely as one reads any essay of Locke, Hume or Kant\index{Kant, Immanuel}, but it should be borne in mind that the stanzas were not intended to be assimilated this way. In fact, we are warned at the very outset by being confronted with the discussion of a preliminary question — ‘who is competent, and consequently entitled, to study the Vedanta in order to realize the truth. The question may be readily answered, so far as we ourselves are concerned: Not we Westerns. Not intellectuals.” 

~\hfill (Campbell\index{Campbell, J.} and Zimmer\index{Zimmer, H. R.} 1956: 51)
\end{myquote}

So then, who is the \textit{adhikārin} (qualified) -and what are the qualities of his mindset? According to \textit{Vedāntaśāra}.

\begin{myquote}
“The “competent student" (\textit{adhikārin}\index{adhikara@\textit{adhikāra}}), when approaching the study of Vedanta, should feel an attitude not of criticism or curiosity, but of utter faith (\textit{śraddhā}\index{sraddha@\textit{śraddhā}}) that in the formulae of Vedanta, as they are about to be communicated to him, he shall discover the truth (\textit{Vedāntasāra}). He must furthermore be filled with a yearning for freedom from the encumbrances of worldly life, an earnest longing for release from the bondage of his existence as an individual caught in the vortex of ignorance." 

~\hfill (Campbell and Zimmer 1956:51)
\end{myquote}

Instead, Pollock\index{Pollock, Sheldon} approaches the text with an ‘attitude of criticism and not with curiosity’ because he analyzes the text with a preconceived theory that he wants to map out combining some past evidence from history (“when Ram temples came into existence”) and the \textit{Rāmāyaṇa}’s\index{Ramayana@\textit{Rāmāyaṇa}} plot (“demonization of Rāvaṇa\index{Ravana@Rāvaṇa}”). Despite his experience with Hindu texts, he fails to acknowledge that in Indian system, beings are divine by nature and everyone is a God in the making, and that \textit{deva}-s and \textit{asura}-s are relative and not absolute.

Some other qualities for a student of \textit{śāstra}-s\index{sastra@\textit{śāstra}} are patience, concentration, and endurance, which a reputed scholar like Pollock may possibly possess; however his open denial of the sacred and lack of \textit{śraddhā}\index{sraddha@\textit{śraddhā}}, restrict him from fully understanding the gist of the \textit{Rāmāyaṇa}. Pollock is placed as an outsider not because he is not an Indian; however, because he does not possess the qualities needed to contextualize the texts in the landscape of Hindu psyche.

\newpage

\begin{myquote}
“Pollock\index{Pollock, Sheldon} also goes against Ingalls, his mentor who stresses dropping the western lens for the study of Sanskrit traditions and the study of \textit{kavya}” 

~\hfill (Malhotra\index{Malhotra, Rajiv} 2016a:L3354)
\end{myquote}

\begin{myquote}
“Ingalls had insisted that Indologists like him must use the Sanskrit tradition’s own lens in studying \textit{kavya}, at least to the extent Westerners were capable.” 

~\hfill (Malhotra\index{Malhotra, Rajiv} 2016a:L3353)
\end{myquote}

A \textit{kāvya}’s purpose, insists Ingalls, is to ‘communicate the \textit{dharma}\index{dharma@\textit{dharma}} to the lay public in a friendly and aesthetically pleasing manner.’(as cited in Malhotra 2016a:L3368). And while, like Pollock, Campbell\index{Campbell, J.} states that the concept of duty in the Occident is different from what it is in the Orient (Campbell 1976:103), he says that it is important to note that a student in the West also does not develop the \textit{śraddhā}\index{sraddha@\textit{śraddhā}} the basic requisite for \textit{guru-śiṣya} tradition. He suggests that what the Orient can instruct the West is in an inward journey – “the mystic inward way into themselves, and this if followed without losing touch with the conditions of contemporary life, might well lead in not a few cases to a new depth and wealth of creative thought and fulfillment in life and in literature and in the arts.” As Malhotra suggests, what Pollock\index{Pollock, Sheldon} ignores is that the very basis of creativity and innovation, is transcending the reality. It is this inward journey that Pollock refuses to take, ignores and shuns, when he separates the \textit{laukika} from the \textit{alaulika}, and refuses to acknowledge how the \textit{pāramārthika}\index{paramarthika sat@\textit{pāramārthika sat}} informs the \textit{vyāvahārika}\index{vyavaharika sat@\textit{vyāvahārika sat}}.

If Pollock did want to look for evidence of how it is that the Indian texts that have allowed for a harmonious political and social life for centuries in India, which neither of two nations that came out of India can claim, he would have mentioned Indian democracy as an intermingling, not only of languages, and cultures but also religions. Pavan Varma\index{Varma, P.}, an Indian diplomat writes:

\begin{myquote}
“Indians do not like the disorder and unpredictability of system-less situations. They are past masters in the art of compromise, in stepping back from the precipice, in forging a modus vivendi that obviates the need to choose between extremes, and in finding solutions that accommodate conflicting interests. Such an approach has sanction of classical notions of statecraft. For instance, according to the \textit{Digvijaya}\supskpt{19} theory of the \textit{Rāmāyaṇa}\index{Ramayana@\textit{Rāmāyaṇa}}, ‘vanquished kings were reinstated in the Kingdom as a matter of principle.” 

~\hfill (Varma 2005:57)
\end{myquote}

Had Pollock been driven by looking at the ground reality, he would have known that in the capital city of India, the burning of Rāvaṇa’s\index{Ravana@Rāvaṇa} effigies every year is a metaphor for change or elimination of social evils, and every year there is a new name for what is called, ‘Aaj Ka Ravan’ (Today’s Rāvaṇa) e.g. inflation, which affects Hindus and Muslims alike. Furthermore, growing up as a Hindu, the author often heard from grandmothers that while the \textit{Mahābhārata}\index{Mahabharata@\textit{Mahābhārata}} was narrated to tell people how they were, it was the \textit{Rāmāyaṇa}\index{Ramayana@\textit{Rāmāyaṇa}} that actually was written to show people how moral, kind, just and self-sacrificing they are capable of being.

Furthermore, like many other scholars Pollock\index{Pollock, Sheldon} has created a language that is not easily accessible to the people whose minds and actions he intends to scrutinize. If his writings cannot be comprehended by the ordinary people he accuses of being so influenced by the ‘political insinuations in the \textit{Rāmāyaṇa}’, then who is Pollock writing for?

\begin{myquote}
“To decode him, one has to read him multiple times. After you understand one theory of his, you need to go back and re-read the prior works you already went through. In places, only after connecting the dots with his other scattered writings can you realize what he wants to say. If his individual points are at times murky, murkier still are the links among the dots to make sense of the big picture. One gets the impression that only a few fellow-travelers subscribing to his ideology are meant to understand him.” 

~\hfill (Malhotra\index{Malhotra, Rajiv} 2016b)
\end{myquote}

Failing to understand his writings, those accused (i.e. those who live the tradition) are ill-equipped to defend themselves and their tradition. Not to mention that Pollock falls prey to the same ‘elitism’ that he has accused Indian texts to generate, when he constantly upholds his thesis above the concept of sacred. In a sense, being an outsider, Pollock is creating boundaries where the very persons who live the traditions, are considered ‘outsiders’ to its understanding.

Pollock’s writings seem to be contributing to atrocity literature more than about understanding Hinduism. And Pollock, an esteemed scholar, does not shy away from using derogatory terms such as, “Temple cult, which is unique to North India.” In mentioning the scarcity of temples dedicated to Rāma before twelfth century, Pollock reads meanings that do not exist in the scarcity of temples dedicated to Rāma\index{Rama@Rāma} because he misunderstands \textit{itihāsa}\index{itihasa@Itihāsa}. ‘History after all occurs in cycles’ (Malhotra 2013). Since documented history of the world is very limited compared to the actual history of the world, the cycles may not have repeated yet. It is quite possible that as we move into Kaliyuga, Durgā worship may take more prominence over Rāma worship, would Pollock\index{Pollock, Sheldon} then come up with a new atrocity that must be related to Durgā worship being against non-Muslims, or may be all men?

It remains for the historians and Indologists alike to analyze the reasons for “the rise of Rāma\index{Rama@Rāma} worship” – was it merely organic, shift in the public consciousness, or was it because the \textit{Rāmāyaṇa}\index{Ramayana@\textit{Rāmāyaṇa}} started to be performed as Rāmlīlā\index{Ramlila@Rām-līlā}? Did it become popular due to certain appreciation of the arts?

However it is clear that Pollock\index{Pollock, Sheldon} is more concerned with forwarding his ideas, creating atrocity theories desperately trying to make links with disconnected events. Malhotra\index{Malhotra, Rajiv} critiques Pollock’s tunnel vision that is so focused on his own conclusions that he generalizes after cherry picking quotes, anecdotes, and sentences from certain texts, and ignores texts like the \textit{Arthaśāstra}\index{Arthasastra@\textit{Arthaśāstra}} which explicitly lays down grounds for the king to look after the welfare of its citizens (Malhotra\index{Malhotra, Rajiv} 2016: L3053). Malhotra quotes the following from the \textit{Rāmāyaṇa} itself to demonstrate that Pollock has deliberately used certain sentences out of context, and linked them with events that have more socio-political reasons to support his ideas.

\begin{myquote}
“As the king so the citizens. Hence, he must lead them by example of his own conduct." 

~\hfill (\textit{Rāmāyaṇa} 2.109.9) (Malhotra 2016:L3062)
\end{myquote}

\vskip 4pt

\begin{myquote}
“Citizens all abiding by \textit{dharma}\index{dharma@\textit{dharma}}, had Rāma as their ideal." 

~\hfill (\textit{Rāmāyaṇa} 6.131.98)
\end{myquote}

On a closing note, the last time I attended ‘Our Rāmlīlā’\index{Ramlila@Rām-līlā}, Ram Uncle had graduated to being the ‘Khewat\supskpt{20}’ the tribal who helped Rāma, Sitā and Lakṣmaṇa cross the Sarayū river. It was an emotional experience to see Ram Uncle now bow to a younger Ram, who is probably more versed in digital media, probably has his own twitter account, and yet, he chooses to take a vow of austerity during the months when he rehearses and performs, and never forgets to bow before he steps on the make-shift stage, for that stage after all is a metaphor for the world where our own stories of love, loyalty, loss and life are played/enacted out.

\newpage

\textbf{Acknowledgements:} The author would like to thank Shri Satchita\-nanda for his expert guidance during the workings of this paper, Dhan Uncle who played King Daśaratha in the Our Rāmlīlā, and shared information on the history and workings of Shri Ram Lila Committee, and Ms. Kavita Sood who became a link between the author and Dhan Uncle.


\section*{Bibliography}

\begin{thebibliography}{99}
\bibitem{chap7-key01} Campbell, J. (1976). \textit{Oriental Mythology: Masks of God}. London: Penguin.

 \bibitem{chap7-key02} —. and Zimmer, H. R. (1956). \textit{Philosophies of India}. New York: Meridian Books.

 \bibitem{chap7-key03} —. Moyers, B. D., and Flowers, B. S. (1991). \textit{The Power of Myth} (1st Ed.). New York: Anchor Books.

 \bibitem{chap7-key04} Christians, C., and Carey, J. (1989). “The Logic and Aims of Qualitative Research”, in Stempel \textit{et al.} (1989).

 \bibitem{chap7-key05} Carpentier, C. (1997) “Mithya, Mythos and Mithal.” From the author’s website. Lecture delivered at the Temenos Academy at a meeting held at The Prince of Wales's Institute of Architecture, Regent's Park, London on 20 October 1997. \url{http://comecarpentier.com/mithya-mythos-and-mithal/. Accessed May 15, 2016.}

 \bibitem{chap7-key06} Embree, A. T. (Ed.). (1972). \textit{The Hindu Tradition: Readings in Oriental Thought}. New York, USA: Random House.

 \bibitem{chap7-key07} González-Reimann, L. (2006). “The Divinity of Rāma in the Rāmāyaṇa of Vālmīki.” \textit{Journal of Indian Philosophy}, 34(3); pp. 203-220.

 \bibitem{chap7-key08} Gould, R. (2008). “How newness enters the world: The methodology of Sheldon Pollock.” \textit{Comparative Studies of South Asia, Africa and the Middle East}, Volume 28, (Number 3). pp. 533- 557.

 \bibitem{chap7-key09} Juluri, V. (2014). \textit{Rearming Hinduism}. Chennai: Westland.

 \bibitem{chap7-key10} Maher, J. M., and Briggs, D. (Ed.s). (1990). \textit{An Open Life: Joseph Campbell in Conversation with Michael Toms}. (1st Ed.) New York: Harper Perennial.

 \bibitem{chap7-key11} Malhotra, R. (2013). \textit{Being Different: An Indian Challenge to Western Universalism}. New Delhi: HarperCollins Publishers India. Kindle Edition.

 \bibitem{chap7-key12} —. (2014). \textit{Indra’s Net: Defending Hinduism’s Philosophical Unity}. Noida, India: HarperCollins Publishers India.

 \bibitem{chap7-key13} —. (2016a). \textit{The Battle for Sanskrit: Is Sanskrit Political or Sacred, Oppressive or Liberating, Dead or Alive?} (1st Ed.). Noida, India: HarperCollins.

 \bibitem{chap7-key14} —. (2016b). “Rajiv Malhotra explains the challenges of understanding Sheldon Pollock.”, \textit{Swarajya}, Online Edition. (April 4, Accessed May 24, 2016).

 \bibitem{chap7-key15} Massey, R. (1992). From Bharata to the cinema: a study in unity and continuity. ARIEL: A Review of International English Literature, 23(1).

 \bibitem{chap7-key16} McCrea, L. (2013). “In the world of men and beyond it: Thoughts on Sheldon Pollock’s the Language of the Gods in the World of Men.” \textit{Comparative Studies of South Asia, Africa and the Middle East}, 33(1). pp. 117-124.

 \bibitem{chap7-key17} Pollock, S. (1984). “The Divine King in the Indian Epic.” \textit{Journal of the American Oriental Society}. pp. 505-528.

 \bibitem{chap7-key18} —. (Trans.) (1991). \textit{The Rāmāyaṇa of Vālmīki: An Epic of Ancient India}, Vol. 3: Araṇyakāṇḍa. (Gen. Ed.) Robert P. Goldman. Princeton: Princeton University Press.

 \bibitem{chap7-key19} —. (1993). “\textit{Rāmāyaṇa} and Political Imagination in India.” \textit{The Journal of Asian Studies}, 52(02); pp. 261-297.

 \bibitem{chap7-key20} —. Elman, B. A., and Chang, K. K. (2015). \textit{World Philology}. Cambridge, Massachusetts: Harvard University Press.

 \bibitem{chap7-key21} Stempel, G. H., and Westley, B. H. (Ed.s) (1989). \textit{Research Methods in Mass Communication}. Englewood Cliffs: Prentice-Hall.

 \bibitem{chap7-key22} Varma, P. K. (2005). \textit{Being Indian: the Truth About Why the Twenty-first Century will be India's}. New Delhi: Penguin Books India.

 \end{thebibliography}

\theendnotes

